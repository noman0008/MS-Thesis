% !TEX root=nonuseinterviews.tex
\section{Introduction}
\label{sec:introduction}
Since their inception in early 2000s, social networking sites and social media have transformed everyday interactive practices. Today, the most popular of these sites, Facebook, claims more than 2 billion global active users. As of 2016, 79\% of Internet users (68\% of all U.S. adults) are estimated to use Facebook~\cite{greenwood2016social}. Such explosive growth and popularity has resulted in a great deal of research attention toward the use and positive impacts of social media in general, and Facebook in particular. For instance, it has been shown that the use of Facebook can be associated with a number of benefits, such as enhancing social connectedness, increasing social capital, and boosting self esteem~\cite{koroleva2011its,ellison2007benefits}.

However, more recent research indicates that increased use of social media may also lead to a number of negative effects, including addiction, feelings of jealousy, depression, decreased well-being, invasion of privacy, reduced work productivity, cyberbullying, etc. As a consequence, people have reported efforts to reduce their use of social media via tactics such as taking a \lq vacation" from Facebook or deleting their accounts. For instance, recent Facebook scandal that exposed large-scale harvesting of its user data by Cambridge Analytica resulted in the trending hashtag \#deleteFacebook.

In the past few years, researchers have recognized the need to investigate such \lq non-use"~\cite{baumer2014refusing}. However, barring a couple of notable exceptions~\cite{baumer2013limiting,lampe2013users}, research on non-adoption, non-use, and abandonment of Facebook has received disproportionately little research attention. Moreover, as Facebook functionality and policies change and people's knowledge and experience regarding Facebook evolve, initial findings regarding non-use may need to be updated correspondingly. To this end, we formulated the following research question: \emph{How can the characterization of non-use and non-adoption of Facebook be refined and extended?}

We addressed the above question via semi-structured interviews with individuals who indicated abandoning Facebook or reducing usage. Many of our findings echo results from the literature, thus boosting their credibility via confirmed replication. Further, we build on these results to contribute a more refined characterization of non-use in the form of various types of \emph{deliberately selective} engagement. Such a characterization provides a more nuanced picture of non-use as covering a range of practices, thus moving beyond a binary portrayal as use or non-use.

In the next section, we summarize the related literature on technology non-use in general and social media non-use in particular. Next, we describe the method we used to tackle our research question along with details of our sample. We proceed to present the results of our analyses in Section~\ref{sec:findings} followed by an application of this insight to suggest potential improvements to Facebook in Section~\ref{sec:implications}. We mention important limitations in Section~\ref{sec:limitations}, and conclude with a highlight of our findings and contributions in Section~\ref{sec:conclusion}.
%In the past few decades, communication within society has changed, mirroring the perpetual increase of social media use~\cite{ellison2007social}. It is theorized that by 2020, 7 out of every 10 US internet users will be connected over social media\footnote{\label{trends}\url{https://contently.com/strategist/2017/04/19/social-media-trends-2017/}}. These platforms allow users to embed free instant messaging and group chats which are typically cheaper than regular cellular text-messaging plans.

%Facebook, launched in 2004, and after being available to the public in 2006, saw an explosion in popularity, reaching 2 billion users in 2015\footnote{\url{https://finance.yahoo.com/news/number-active-users-facebook-over-years-214600186--finance.html}}. It is now the most popular social media website. Every 60 seconds, there are 510 comments written, 293,000 statuses made, and 136,000 photos uploaded on Facebook~\cite{greenwood2016social}. As of 2016, 79\% of internet users (68\% of all U.S. adults) use Facebook~\cite{greenwood2016social}. As other social media websites, like Instagram, Snapchat, and Twitter have seen significant growth, Facebook market survival strategies have included updates to various aspects of privacy control. These aspects include the website's interface, the company's privacy policies, the way the users interact with the site, among others. One of the reasons social media has become an integral part of so many people's lives is because it provides the ability to self-represent by creating an online presence through content production. Ellison et al. noted that social media websites and platforms help in developing relationships and connections in both personal and professional aspects of their life~\cite{ellison2007benefits}.

%Despite the many benefits of connecting over social media, users often experience extreme pressure to join social media platforms either in their personal life or under pressure by external entities such as their workplace. Being a part of a virtual community is ubiquitous and people who abstain may be seen as misanthropes. In spite of its popularity, Facebook scores low in customer satisfaction surveys, with users reporting issues with privacy, interface changes, and spam. Gillette mentioned that Facebook usage has been slowly declining recently, as users have been shifting to other social media platforms~\cite{gillette2015facebook}. Facebook has received much criticism in regards to its handling of user data and privacy\footnote{\url{https://www.dailydot.com/via/killing-facebook-dependency-monopoly/}}. Sophos mentioned that businesses see Facebook as the most dangerous social media website for user security, with 60\% of businesses believing that it is more dangerous than MySpace, Twitter, and LinkedIn\footnote{\url{https://www.sophos.com/en-us/security-news-trends/security-trends/network-security-top-trends.aspx}}. Criticisms about Facebook's usage of user data has been harsh, with the platform previously being ranked in the bottom 5\% of private sector companies in user satisfaction~\cite{greenwood2016social}. In 2011, the American Customer Satisfaction Index listed Facebook as having a score of 64 out of 100, putting it in the same range as the IRS's e-filing system, airline websites, and cable companies~\cite{guynn_2014}. Researches have indicated that users are getting concerned about the negative effects of social media and how it is degrading their physical and mental health~\cite{newyorktimes2017}. 

%The delete Facebook hashtag (\#DeleteFacebook) has become popular on online portals~\footnote{\url{https://www.cnet.com/news/deletefacebook-hashtag-trends-twitter-facebook-users/}}, even moreso after the news of the Cambridge Analytica data breach~\footnote{\url{https://www.nytimes.com/2018/03/19/technology/facebook-cambridge-analytica-explained.html}}. Though these data breaches create privacy doubts among users and affect the trust they have on the company, a correlation is yet to be proved with the privacy and security concerns of users with their abandonment of Facebook. Previous research and articles provide instances where users have experienced negative opinions about social media as an \textit{agent of chaos} despite the many social benefits it provides.

%Studies about a system non-use or failure is not only important in learning about the decline of a system but also to analyze how and why we can improve a user experience to cater to their requirements, since non-usage is extensively related to the user behavior rather than just being limited to a system or technology at all~\cite{baumer2011implication, selwyn2005whose, selwyn2003apart}. As mentioned by Baumer et al. the study of non-use of technology specially that of social media platforms such as Facebook indicate the representation of one's social identity~\cite{baumer2015study} and we aim in unravelling the nuances of one's identity with their social representation of self. While much of the current research explores why internet users choose Facebook as their preferred social media platform, our goal was to find out why users may choose to limit their usage or abandon the website altogether and whether Facebook will follow in the footsteps of the many social media giants before them, such as MySpace. In the light of these discussions, our study focuses on answering the following research questions:\\
%\textit{RQ1. What are the factors that contribute to non-use and disengagement from social media?}\\
%\textit{RQ2. What effects do privacy and security concerns have on an individual's social media behavior?}\\
%\textit{RQ3. What are the various levels of disengagement users have on social media?}\\

%We answered the above research questions by conducting a pre-screening survey, where 248 participants responded. Out of the 248 responses, we selected 22 participants for the interviews who specifically mentioned they do not have Facebook account or have curtailed their usage since last year. Based on our analysis, we make the following contributions. We proved that binary or tertiary classification of usage of systems provides overall archival reasons for dissatisfaction. However, to understand the negative sentiments of social media, we develop specific classification for non-usage.

%We uncovered the reasons why users stay connected virtually and choose to not produce any content themselves. We also provide a direct connection of how social media modifications indicate such changes, with personal changes possibly being the causative agent. Privacy and security concerns though often mentioned by many users is not always reflected in one's usage and non-usage. Alternatively, a reduction of use or to be remained unseen has been a trend noted which can . We also developed a comprehensive coding scheme of the detailed interviews to create a list of reasons that can be a primary cause of user's behavior online. We classify the usage based on the selective or detached engagement of the users including feature based usage and audience based usage. To find a balance between user experience, perception, and technical feasibility, we make recommendations to reduce negative sentiments of a user.

%In the following section, we summarize prior research on the theorized reasons for user dissatisfaction with social media websites, often leading to the reduction in usage, if not abandonment of the platform. We then outline the method in Section~\ref{sec:method} applied for our study along with the details of participant recruitment and a description of the participant set. Next, we describe our findings and discuss about the insight that emerged in the Section~\ref{sec:findings}. We proceed to apply the insight to suggest a number of potential improvements to Facebook, the accompanying user experience, and other related aspects in the Section~\ref{sec:implications}. We point out important limitations and avenues for future work in section~\ref{sec:limitations} and conclude in section~\ref{sec:conclusion}.
% !TEX root=nonuseinterviews.tex
\section{Method}
\label{sec:method}
To address the research question outlined in Section~\ref{sec:introduction}, we conducted semi-structured interviews with individuals who had indicated not having a Facebook account or reducing Facebook usage in the preceding year. We recruited participants between December 2016 to May 2017 using a short screening questionnaire (see Appendix A) distributed via mailing lists, online forums, social media, local Craigslist site, and flyers posted around campus. To avoid priming, we advertised the research as a study of online communication practices without revealing our focus on social media non-use. The screening questionnaire asked for basic demographics along with social media use.

From the 248 people who responded to the screening questionnaire, we selected those who indicated not having an account on Facebook or reducing their use of Facebook during the previous year. To limit the impact of cultural factors, we further narrowed this set to include only native English speakers over the age of 18 who indicated having lived in the US for at least 5 years. Ultimately, we conducted semi-structured interviews with 22 of these individuals. Table~\ref{tab:participants} shows the age, gender, and occupation of the participants. As Table~\ref{tab:participants} indicates, 5 participants did not have a Facebook account at the time of the interview while 16 reported reducing their usage in the preceding year. The remaining participant (P10) had re-activated her Facebook account which was deactivated at the time she responded to the screening questionnaire. While a large proportion of our participants are students, they cover a wide variety of fields, such as Athletics, Business, Informatics, Music, Psychology, Speech Sciences, etc. Further, the overall sample is balanced in terms of gender and covers a wide age range (19--44).

We conducted 1-1 semi-structured interviews with each of the 22 participants in person. The interviews lasted about 30--45 minutes and followed a conversational style covering a broad range of topics related to various aspects of their social media use and non-use (see Appendix B). In addition to taking notes during the interviews, we audio recorded and transcribed the interviews for analysis. Participants were compensated US \$10 for taking part in the study. All study procedures were approved by our university's Institutional Review Board (IRB).

%\begin{table}[!htb]
%    %\caption{Global caption}
%    \begin{minipage}{.45\linewidth}
%      %\caption{}
%      \centering
%        \begin{tabular}{ |p{0.5cm}||p{0.5cm}|p{0.5cm}|p{2cm}|p{0.6cm}|}
%\hline
% P & Age & Sex & Education/ Occupation & Fb (Y/N)\\
% \hline
% P1&25&M&Staff&Y\\ 
% P2&25&F&Faculty&Y\\ 
% P3&23&F&Staff&Y\\ 
% P4&24&F&Graduate&Y\\ 
% P5&20&F&Undergraduate&Y\\ 
% P6&44&-&Graduate&Y\\ 
% P7&23&M&Undergraduate&Y\\ 
% P8&20&F&Undergraduate&Y\\ 
% P9&22&M&Undergraduate&N\\
% P10&23&F&Undergraduate&N*\\
% P11&21&M&Undergraduate&Y\\ 
%\hline
%\end{tabular}
%    \end{minipage}
%    \begin{minipage}{.45\linewidth}
%      \centering
%        %\caption{}
%        \begin{tabular}{ |p{0.5cm}||p{0.5cm}|p{0.5cm}|p{2cm}|p{0.6cm}|}
%        \hline
% P & Age & Sex & Education/ Occupation & Fb (Y/N) \\
%\hline
% P12&32&M&Service&N\\ 
% P13&23&F&Undergraduate&Y\\ 
% P14&22&F&Undergraduate&Y\\ 
% P15&19&M&Undergraduate&Y\\ 
% P16&22&M&Undergraduate&Y\\
% P17&37&F&Graduate&Y\\
% P18&22&M&Undergraduate&Y\\
% P19&21&F&Undergraduate&Y\\
% P20&29&M&Graduate&N\\
% P21&34&M&Staff&N\\
% P22&28&M&Service&N\\
% \hline
%\end{tabular}
%    \end{minipage} 
%    \caption{Demographics of the Sample of Study Participants.}
% \label{tab:participants}
%  \vspace{-5mm}
%   \vspace{-5mm}
%\end{table}

\begin{table}[!htb]
   
      \centering
        \begin{tabular}{ |p{1cm}||p{1cm}|p{1cm}|p{2.5cm}|p{1cm}|}
\hline
 P & Age & Sex & Education/ Occupation & Fb (Y/N)\\
 \hline
 P1&25&M&Staff&Y\\
 P2&25&F&Faculty&Y\\
 P3&23&F&Staff&Y\\
 P4&24&F&Graduate&Y\\
 P5&20&F&Student (UG)&Y\\
 P6&44&-&Student (G)&Y\\
 P7&23&M&Student (UG)&Y\\ 
 P8&20&F&Student (UG)&Y\\
 P9&22&M&Student (UG)&N\\
 P10&23&F&Student (UG)&Y\\
 P11&21&M&Student (UG)&Y\\
 P12&32&M&Physicist&N\\ 
 P13&23&F&Student (UG)&Y\\ 
 P14&22&F&Student (UG)&Y\\
 P15&19&M&Student (UG)&Y\\ 
 P16&22&M&Student (UG)&Y\\ 
 P17&37&F&Student (G)&Y\\ 
 P18&22&M&Student (UG)&Y\\
 P19&21&F&Student (UG)&Y\\ 
 P20&29&M&Student (G)&N\\
 P21&34&M&Staff&N\\
 P22&28&M&Social Worker&N\\ 
\hline
\end{tabular}
    \caption{Demographics of the Sample of Study Participants.}
 \label{tab:participants}
  \vspace{-5mm}
   %\vspace{-5mm}
\end{table}
%P1: N/A
%P2: N/A
%P3: N/A
%P4:Psychological & Brain Sciences
%P5:Education
%P6:Vice Pres Information Technology
%P7:Jacobs School Of Music
%P8:Media
%P9:Athletics
%P10:Marketing
%P11:Accounting and Finance
%P12: N/A
%P13:Sales and Marketing Education
%P14:Media
%P15:Marketing
%P16:Financial Accounting
%P17:Informatics(Computing, Culture, and Society)
%P18: Finance and Economics
%P19: Operations Management, Marketing
%P20:Speech & Hearing Sciences
%P21: N/A
%P22: N/A

%\subsection{Interview}
%The interview script was vague, giving the researcher the ability to improvise questions throughout the repeated pilot studies. Please see the semi-structured interview questions in the supplementary materials for more details.

%The varied range of questions for analyzing the online behavioral pattern of the users included questions about other social media platforms apart from Facebook, such as: "What other social media platforms do you use?", "Do you use your Facebook account to log into other services on the Internet? (Why or why not?)", etc. Furthermore, the participants who had been using their Facebook account up until the interview were asked questions such as: "What prompted you to create a Facebook account?", ''How long have you had a Facebook account?", etc. We wanted to know why the participants used Facebook, so we asked questions such as- "What aspects of using Facebook do you like/dislike?", "Which features of Facebook do you typically use (i.e., chat, groups, events, etc.)? Why?"

%Through our set of questions, we wanted more close insight on how the participants manage their online presence through their Facebook account, thus we asked questions such as: "What do you think people think of you based on your Facebook profile?
%Have you ever posted anything that you later edited or deleted? Why?"
%We also wanted to know why users choose to reduce usage or "lurk" on Facebook, so we asked questions such as: Which of the two modes (producer and consumer) is your predominant or preferred way of using Facebook? Why?"
%Our study was primarily focused on uncovering whether privacy plays a role in making the users decide what to post or what to leave out, we asked questions such as: "In what ways is your use of Facebook ''public"?
%What are your privacy settings on Facebook (public, friends only, only me, others) and why?
%Those who didn't have a Facebook account were asked questions such as- Did you ever have a Facebook account?
%If Yes, why did you choose to get rid of the Facebook account?
%If No, why did you choose not to create a Facebook account?

%The interviews were one-on-one interviews and took 30-45 minutes to complete. The participants were given \$10.00 cash each as a token of our appreciation. As per the University Review Board's regulations, we promised to pay the participant the compensation amount whether or not they chose to complete the study. Fortunately, none of our participants chose to drop the study mid-way and were highly interested in sharing their social media experience with us through interviews. 
\begin{table*}[htb!]
\begin{tabular}{ |p{3cm}||p{4cm}|p{10cm}| }
 \hline
 Themes & Underline Categories & Example Quotes from the Interview\\
 \hline \hline
 Audience Influence & Friends Only, Older People, Close-Knit, Friends Of Friends, Impression Management&P7: \textit{I think it was gradual. I enjoy producing content now as a writer producing plays in my free time. I think Facebook is not the type of audience I wanted for what I do. I'm not a high artist or anything like that.}\\
 \hline
 Influence of External Factors &School, Job, Everyone Has, Social Influence, Forced to talk& P10: \textit{I didn't have one for 6 months in the Fall, because I was part of an organization where we couldn't have any social media, but it ended in January, but it was like before that I had one and now I have one.} \\
 \hline
 Facebook Features &Advertisement, Facebook Modifications, Ease of Use, Facebook Games, Messenger& P7: \textit{I think that a lot of it is advertising of course, I do not like. You get the little banner ads on the side.}\\
 \hline
 Modified Behavior & Hide Posts, Passive, Reduced Usage, More Responsible, Infrequent Use& P13: \textit{I would say I post less often, just cause I don't get to know if people get to see a bunch of your posts.} \\
 \hline
 Negative Perceptions of Facebook    & Alcohol, Creepy, Blocked, Anger, Abandon&P19: \textit{Obviously using curse words, um, content that insinuates drinking or illegal activities, that type of stuff, um, things that a lot of people have conflicts over, so for something like politics, everybody has their view on politics and a lot of people don't agree, so I wouldn't post something on Facebook.}  \\
 \hline
 Non-Facebook Media Use   & Other Social Media, Migrating, Other Media, Telephone& P14: \textit{I feel like Facebook is one of the more soc- like one of the social media sites that young people and older people use, because if you go on Instagram and Snapchat and Twitter, they're more for like, the younger generations, so that's why I use it.}\\
 \hline
 Personal Acumen    & Do Not Invest Time, Do Not Read Policies, Boredom, Age, Procrastinate& P2: \textit{Teenage years tend to be a little bit more interesting. People tend to overshare a little bit more. So when I was younger, yes, I did post a lot more with status updates, thinking that people cared about what you post. And at this point in time I don't really care to share as much. }\\
 \hline
 Privacy and Security Concerns  & Privacy Concern, Public, Sell Information, Uses Data, Turn Off Chat& P20: \textit{They(Facebook) want to probably make a lot of money, they use my data to probably track where, what sites I visit or what types of posts I subscribe to, they want to maybe know my spending habits, the types of movies I watch, they probably analyze the entire thing, for their benefit, um… and they just want to keep me as their customer, as a consumer, so that they can keep their billions.}\\ 
  \hline
 \hline
\end{tabular} 
\caption{Table of Code Themes with examples quotes associated with the underline categories}
 \label{table:coding}
\end{table*}

%\subsection{Transcription and Coding}
Interview transcripts and notes were analyzed using the RQDA\footnote{\url{http://rqda.r-forge.r-project.org/}} package in R. The analysis utilized the techniques of Grounded Theory~\cite{glaser1968discovery} with initial open coding followed by axial and selective coding. Open coding resulted in 277 codes. We used axial coding for linking and relating the codes and selective coding for thematic integration. The following section describes the results of our analyses.
%\subsubsection{Open-Coding}
%We used open-coding to explore all the avenues of the social media usage of the participants. The RQDA \footnote{\url{http://rqda.r-forge.r-project.org/}}, package of R was used for coding the transcribed data. We followed the ethnographic coding scheme of open, axial, and selective coding ~\cite{glaser1968discovery} \footnote{\url{http://euroac.ffri.hr/wp-content/uploads/2011/07/Data-Analysis_Coding.pdf}}$^{,}$\footnote{\url{http://changingminds.org/explanations/research/analysis/ ethnographic_coding.htm}}. Open-coding of the semi-structured interviews helped the first author in labeling and defining the categories based on the interview data, without any prior assumptions of the responses. During open-coding, all of the responses were coded word-for-word to avoid any loss of data. For example, if the participant mentioned- \textit{I am sometimes creeped out. If I search something, it pops out in my feed. I know they pay money to Facebook to feature their ad. I am not majorly concerned but sometimes it creeps me out.}- this particular sentence was multi-coded to advertisement and privacy concern. The reason this particular response fell under both categories was because advertisement was seen as one of the major concerns of the participants. The open coding generated 277 unique codes. After coding all the responses, the researcher moved to Axial coding, which generated a relationship and connection between all 277 codes.
%\subsubsection{Selective Coding}
%The open coding gave 277 underline coding, thereafter we found code connections with axial coding, probing more on selective coding. This was both helpful for the analysis and enabled us in analyzing all the details provided by the participants via the interviews. The coding was done by the first author of the paper, thereafter the verifier observed the coding strategies and the analysis. For selective coding, we categorized them into 10 themes; the themes indicated both the factors which lead to non-use of Facebook such as Audience or Influence of External Factors and also effects such as No Facebook or Modified Behaviour. The reason the coding was done as such was to segregate the non-users from the inactive users. We also wanted to learn the reasons why some users decreased their usage over the last year.
%We chose underline categories group together to create the themes, for example: Do not comment (on others posts), Edited(personal content), Tag(others or getting tagged), Track Others( tracking others, not actively posting but reading about others) were all categorized under the theme-Activity. The details of the categories is provided in Table~\ref{table:coding}. 
%The following section discussed on the high level findings we derived from the interviews. 
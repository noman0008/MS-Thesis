% !TEX root=nonuseinterviews.tex
\section{Related Work}
\label{sec:relatedwork}
The Internet is shaping how we communicate with each other~\cite{wellman2003social} and social media paved the road of communication~\cite{bijker2012social}. Facebook is the largest social networking website and despite criticisms, its usage continues to grow~\cite{joinson2008looking}. However, we notice users concerns regarding various features of Facebook, including its data usage policy, often moving to other social media platforms?, privacy concerns regarding posted content, advertisements, and others. These concerns lead to reduced posting, increased lurking, self-censorship, and even abandoning the platform all together~\cite{wyatt2003non,karppi2011digital,gillette2015facebook}. The creation of a Facebook quit day, where people commit to quit Facebook and never return is an example of these practices\footnote{\url{http://www.quitfacebookday.com}}. 

Technological non-use has been a matter of concern for Human Computer Interaction (HCI) scholars as it not only focuses on the design changes, it also focuses on the user experience as a whole which is a critical predictor of improved security adoption and use~\cite{baumer2015importance}. Though much work is being done on usage of social media, there are essential literature gaps in understanding the dissatisfaction in consumer base, specifically of Facebook, through qualitative research. This related work section aims to provide a comprehensive analysis of the user cited reasons through different works expressing concerns on the usage of social media, especially Facebook.

\subsection{Lurkers and Non-Usage}
Lurking is defined and interpreted differently by various researchers~\cite{crawford2009following,schultz2004lurkers} and while discussing non-use it is very important not only to discuss usage versus non-usage but also to focus on users curtailing their usage in different levels. For our study, we focus on lurkers who have curtailed their usage specifically on Facebook. We also address users who have reduced their usage and abandoned the platform entirely.

Previous studies into Facebook lurking, reduced usage, and non-usage found that there are varied contributors that lead to limiting the use of Facebook to a minimal level, deactivation of accounts, and deletion of accounts. While different factors drive these changes in the user base, one common factor which was present among the users in this study were concerns that they have regarding their privacy on Facebook~\cite{baumer2013limiting}. Nonnecke et al.~\cite{nonnecke2001lurkers}, Preece et al.~\cite{preece2004top}, and Birnholtz~\cite{birnholtz2010adopt} explored the reasons behind lurkers studying several groups over Internet through interviews, however Facebook is a larger community and warrants separate study since Facebook abandonment can arise from specific features including advertisements, moderated news feed, email notifications, etc. These studies, however, while similar in subject matter and methodology differ in the approach as well as the granularity of non-usage. 

\subsection{Privacy Concerns}
Users have become increasingly aware of privacy concerns~\cite{hargittai2010facebook}, specifically in the post-Snowden era~\cite{rainie2015americans}, and more users have switched their account to Friends Only or Custom Settings compared to the first few years of Facebook~\cite{fuchs2012political,madden2012privacy}. This shift might also be explained by a lack of granular privacy settings earlier in Facebook's history as opposed to newer privacy controls. More than half of social media users (58\%) restrict access to their profiles by customizing the privacy settings, especially women~\cite{madden2012privacy}. We noted similar results as discussed in the Section~\ref{sec:findings}.

Participants in our study also mentioned that with age the amount of disclosure reduced, similar to Nosko et al.'s work~\cite{nosko2010all}. Lampe et al. argue that privacy plays a vital role in users not joining Facebook~\cite{lampe2013users} and even leading to commit virtual (social media) suicide by leaving the platform~\cite{stieger2013commits}. Perceived privacy and security, though might be different from the actual practices preached, plays a vital role in user interaction over social media~\cite{jung2016imagined,shin2010effects,debatin2009facebook,liu2011analyzing}. These concerns are enhanced for romantic partners whose Facebook experience is made more difficult due to  omnipresence of data and the ability to track a partner's posts~\cite{gershon2011friend,madden2006online,zhao2012s}. Stories from families and friends about privacy and security concerns also play an influential role in reduction of usage by Facebook users~\cite{rader2012stories}. 

Privacy and security concerns vary among different individuals indicating privacy inequalities~\cite{litt2013understanding,kang2015my} and can have important implications in design of a system~\cite{dupree2016privacy}. What is practised and preached in privacy and security concerns vary greatly ~\cite{phelan2016s} and it becomes difficult for researchers and developers to understand the details with such granularity and specificity. As a result, a binary classification as use vs.\ non-use is unable to fully characterize non-use practices of individuals. With our study, we aim at clustering various concerns leading to different or similar attitudes and behaviors of social media users and generate a typology of non-use.

\subsection{Monitoring Old Habits}
Bauer et al., found that users desired to have the ability to go through old posts as well as modify or enhance their privacy settings ~\cite{bauer2013post}. This, often referred to as monitoring, indicates that users want to represent themselves in a way that their past posts do not justify. Self-representation is extremely important for social media users~\cite{dimicco2007identity,zhao2008identity}, thus monitoring their virtual image is of high importance as well. Our analysis, as explained in Section~\ref{sec:findings} indicates similar patterns among the users where they have expressed interest in a better way to monitor their posts and sometimes delete it.

\subsection{Audience and Other Social Media}
The audience of social media, especially Facebook, is transitioning from the younger generation to older individuals which contributes to the non-usage of Facebook. The inter generational social media usage difference is increasingly blurring~\cite{bucur1999older}. Bauer et al. suggests that users may be leaving sites like Facebook in favor of newer platforms like Snapchat. On Snapchat, a photo stays in the story only for 24 hours, and the user is notified if another user takes a screen shot of the photo~\cite{bauer2013post}. We noticed similar reactions from some of our participants as well, thus, indicating some users do not like the permanency of the posts made on Facebook. 
%However our results signify that this depends on several factors. A participant expressed using Facebook as personal media storage, indicating positive implication of such a feature.

\subsection{Abusive Posts}
Mental and physical health is often considered to be a primary source of concern. Social media, though purportedly just a medium to connect individuals, has often been noted to affect users negatively, sometimes due to usage time and sometimes due to abusive content~\cite{newyorktimes2017}. Our research showed that users, including supporters of free speech, often expressed dissatisfaction regarding abusive posts, especially regarding religious or political sentiments. Adults are however less affected as compared to teens~\cite{lenhart2011teens,rainie2012tone} by such posts, but the abusive nature of the content not only contributes to bullying over the Internet, but also leads to the discontinuation of the use of such sites.

\subsection{Social Influence}
Non-Use though a personal choice can be influenced positively or negatively by various other social and behavioral factors. Baumer et al. used a survey to study how several factors including, fear of missing out, reaction of one's social media audience impact one's perspective of not being able to leave a system~\cite{baumer2015missing}. Though survey techniques  provide an overall archival reasons for non-usage of social media platforms it does not provide the detailed analysis of what prevents one from completely abandoning such platforms. As mentioned by Baumer et. al's paper, every method including surveys to machine learning including qualitative and quantitative techniques have unique approaches to a similar problem providing more and detailed analysis of a problem~\cite{baumer2017comparing}.

Baumer et. al's other studies on details of deactivation of Facebook and reduction of usage of social media provide a mixed-methods view based on surveys to interviews. However, it demarcates the usage between group communication and whether people chose to deactivate or not~\cite{baumer2017subjects}. One's usage of social media is not binary (use and non-use) or ternary (use, limited use, and non-use). Instead, non-use may cover a wide spectrum of practices involving various levels of engagement on social media.
%Squirell in this article mentions how trolls are leading to hate speeches and even frog memes are one of the most rated hate symbols ~\cite{squirrell_2017}. 

\subsection{Miscellaneous Reasons for Lurking}
Though studies on non-usage have not been limited to Facebook, but have also covered other social media platforms, such as Twitter~\cite{schoenebeck2014giving}, varying levels of (non-)usage is still understudied. Ames et al. identified the negative effect of multitasking as one of the primary reasons for lurking on social media~\cite{ames2013managing}. Whereas, Portwood argued that the fear of misinterpretation is one of the key reasons for reducing usage on Facebook~\cite{portwood2013media}. There are arguments about the digital divide and how it prevents one from communicating similarly to a privileged individual, contributing to the avoidance of social media all together~\cite{van2005deepening,warschauer2004technology}. Hargittai explores more on the social divide by mentioning that people with more technical expertise are likely to use social media~\cite{hargittai2007whose}. Ryan and Xenos instead try to analyze the characteristic traits of individuals who use Facebook, thus indicating the negative traits by exploring the Big 5 characteristics~\cite{ryan2011uses}. Verdegem and Verhoest grouped all the traits for non-use together, mentioning inaccessibility, lack of skills, and negative perception of the technologies to be the key reasons to reduce usage~\cite{verdegem2009profiling}. Design issues have also been considered in making users cease usage of Web site~\cite{pierce2012undesigning,satchell2009beyond}. All these reasons though effective is highly scattered across various facets of social media. We apply our findings to make design recommendations for Facebook to improve its user experience (see Section~\ref{sec:implications}).

%Title and abstract
\begin{comment}
1. FACEBOOK’S FALL IN THE SOCIAL MEDIA AGE
Almost from its inception, the social media universe was centered around one all-encompassing network, Facebook. Today, with technological advancements and an experienced audience of social networkers, Facebook is struggling to meet the increasingly sophisticated and complex array of demands from its users. That is primarily because social networking is now a mass media. Networks that focus on niche markets and specific uses are more effectively addressing the needs of specific groups and are therefore becoming more popular. Facebook is still the largest social network by size, but that is not enough to guarantee its future.

This project is based on my analysis of reports conducted in the field of social networking from academic articles and reports by think tanks and market research firms. To understand the current sphere of social media I analyzed users and their habits. This includes who uses social networks, how they are accessing them, how often and why. I also analyzed research describing the growth and populations of the largest and fastest growing social networks. I conducted a series of interviews with academics in the field of social networking to gain a further understanding of today's social media market and the future of social networking.

Through this systemic view of how people use social networks we can understand the market forces influencing and altering social media. Informed by this research I discovered a hole in the current social media market. Therefore, I have also created the framework for a social networking app to fill this void.

2. https://www.statista.com/statistics/264810/number-of-monthly-active-facebook-users-worldwide/

3. http://www.pewinternet.org/2016/11/11/social-media-update-2016/

4. https://www.sophos.com/en-us/security-news-trends/security-trends/network-security-top-trends.aspx

5. https://www.usatoday.com/story/tech/2014/07/22/facebook-customer-satisfaction/12994539/

6. Users and nonusers: interactions between levels of adoption and social capital

Although Facebook is the largest social network site in the U.S. and attracts an increasingly diverse userbase, some individuals have chosen not to join the site. Using survey data collected from a sample of non-academic staff at a large Midwestern university (N=614), we explore the demographic and cognitive factors that predict whether a person chooses to join Facebook. We find that older adults and those with higher perceived levels of bonding social capital are less likely to use the site. Analyzing open-ended responses from non-users, we find that they express concerns about privacy, context collapse, limited time, and channel effects in deciding to not adopt Facebook. Finally, we compare non-adopters against users who differ on three dimensions of use. We find that light users often have social capital outcomes similar to, or worse than, non-users, and that heavy users report higher perceived bridging and bonding social capital than either group.

7. Limiting, leaving, and (re)lapsing: an exploration of facebook non-use practices and experiences

Despite the abundance of research on social networking sites, relatively little research has studied those who choose not to use such sites. This paper presents results from a questionnaire of over 400 Internet users, focusing specifically on Facebook and those users who have left the service. Results show the lack of a clear, binary distinction between use and non-use, that various practices enable diverse ways and degrees of engagement with and disengagement from Facebook. Furthermore, qualitative analysis reveals numerous complex and interrelated motivations and justifications, both for leaving and for maintaining some type of connection. These motivations include: privacy, data misuse, productivity, banality, addiction, and external pressures. These results not only contribute to our understanding of online sociality by examining this under-explored area, but they also build on previous work to help advance how we conceptually account for the sociological processes of non-use.

8. The post anachronism: the temporal dimension of facebook privacy

This paper reports on two studies that investigate empirically how privacy preferences about the audience and emphasis of Facebook posts change over time. In a 63-participant longitudinal study, participants gave their audience and emphasis preferences for up to ten of their Facebook posts in the week they were posted, again one week later, and again one month later. In a 234-participant retrospective study, participants expressed their preferences about posts made in the past week, as well as one year prior. We found that participants did not want content to fade away wholesale with age; the audience participants wanted to be able to access posts remained relatively constant over time. However, participants did want a handful of posts to become more private over time, as well as others to become more visible. Participants' predictions about how their preferences would change correlated poorly with their actual changes in preferences over time, casting doubt on ideas for setting an expiration date for content. Although older posts were seen as less relevant and had often been forgotten, participants found value in these posts for reminiscence. Surprisingly, we observed few concerns about privacy or self-presentation for older posts. We discuss our findings' implications for retrospective privacy mechanisms.

9. https://qz.com/1056319/what-is-the-alt-right-a-linguistic-data-analysis-of-3-billion-reddit-comments-shows-a-disparate-group-that-is-quickly-uniting/

10. https://www.nytimes.com/2017/08/18/opinion/sunday/technology-downgrade-sanity.html

11. The Benefits of Facebook “Friends:” Social Capital and College Students’ Use of Online Social Network Sites

This study examines the relationship between use of Facebook, a popular online social network site, and the formation and maintenance of social capital. In addition to assessing bonding and bridging social capital, we explore a dimension of social capital that assesses one’s ability to stay connected with members of a previously inhabited community, which we call maintained social capital. Regression analyses conducted on results from a survey of undergraduate students (N = 286) suggest a strong association between use of Facebook and the three types of social capital, with the strongest relationship being to bridging social capital. In addition, Facebook usage was found to interact with measures of psychological well-being, suggesting that it might provide greater benefits for users experiencing low self-esteem and low life satisfaction.

12. https://prpost.wordpress.com/2013/07/22/an-example-of-how-to-perform-open-coding-axial-coding-and-selective-coding/

13. The discovery of grounded theory; strategies for qualitative research

Book

14. Privacy management on social media sites

No abstract
http://www.isaca.org/Groups/Professional-English/privacy-data-protection/GroupDocuments/PIP_Privacy%20mgt%20on%20social%20media%20sites%20Feb%202012.pdf

15. Following you: Disciplines of listening in social media
This paper develops the concept of listening as a metaphor for paying attention online. Pejorative terms such as ‘lurking’ have failed to capture much detail about the experience of presence online. Instead, much online media research has focused on ‘having a voice’, be it in blogs, wikis, social media, or discussion lists. The metaphor of listening can offer a productive way to analyse the forms of online engagement that have previously been overlooked, while also allowing a deeper consideration of the emerging disciplines of online attention. Social media are the focus of this paper, and in particular, how these platforms are changing the configurations of the ideal listening subject. Three modes of online listening are discussed: background listening, reciprocal listening, and delegated listening; Twitter provides a case study for how these modes are experienced and performed by individuals, politicians and corporations.

16. The top five reasons for lurking: improving community experiences for everyone

Even in busy online communities, usually only a small fraction of members post messages. Why do so many people prefer not to contribute publicly? From an online survey that generated 1,188 responses from posters and lurkers from 375 MSN bulletin board communities, 219 lurkers spoke out about their reasons for not posting. While lurkers did not participate publicly, they did seek answers to questions. However, lurkers’ satisfaction with their community experience was lower than those who post. Data from 19 checkbox items and over 490 open-ended responses were analyzed. From this analysis, the main reasons why lurkers lurk were concerned with: not needing to post; needing to find out more about the group before participating; thinking that they were being helpful by not posting; not being able to make the software work (i.e., poor usability); and not liking the group dynamics or the community was a poor fit for them. Two key conclusions were drawn from this analysis. First, there are many reasons why people lurk in online discussion communities. Second, and most important, most lurkers are not selfish free-riders. From these findings, it is clear that there are many ways to improve online community experiences for both posters and lurkers. Some solutions require improved software and better tools, but moderation and better interaction support will produce dramatic improvements.

17. Why lurkers lurk
The goal of this paper is to address the question: ëwhy do lurkers lurk?í Lurkers reportedly makeup the
majority of members in online groups, yet little is known about them. Without insight into lurkers, our
understanding of online groups is incomplete. Ignoring, dismissing, or misunderstanding lurking distorts
knowledge of life online and may lead to inappropriate design of online environments.
To investigate lurking, the authors carried out a study of lurking using in-depth, semi-structured interviews with
ten members of online groups. 79 reasons for lurking and seven lurkersí needs are identified from the interview
transcripts. The analysis reveals that lurking is a strategic activity involving more than just reading posts.
Reasons for lurking are categorized and a gratification model is proposed to explain lurker behavior.

18. Managing Mobile Multitasking: The Culture of iPhones on Stanford Campus
This paper discusses three concepts that govern technosocial practices among university students with iPhones. First is the social expectation of constant connection that requires multitasking to achieve. Second is the resulting technosocial pecking order of who gets interrupted or ignored for whom. Third is the way that many students push back against these demands with techno-resistance, deliberately curtailing constant connection to reduce the negative effects of multitasking, in spite of the risk of social censure. These concepts are developed from interviews with 57 students, 30 hours of field observations, and a survey of 177 students on Stanford campus, which in particular explored iPhone use. This research concludes that so-called "digital natives" must still navigate familiar social dynamics and personal desires, both online and off. Providing a detailed description of how students from across campus make sense of iPhones in their everyday technosocial assemblages, this research suggests opportunities for more socially and cognitively sensitive design of smartphone features.

19. Media refusal and conspicuous non-consumption: The performative and political dimensions of Facebook abstention
This paper is a study of consumer resistance among active abstainers of the Facebook social network site. I analyze the discourses invoked by individuals who consciously choose to abstain from participation on the ubiquitous Facebook platform. This discourse analysis draws from approximately 100 web and print publications from 2006 to early 2012, as well as personal interviews conducted with 20 Facebook abstainers. I conceptualize Facebook abstention as a performative mode of resistance, which must be understood within the context of a neoliberal consumer culture, in which subjects are empowered to act through consumption choices – or in this case non-consumption choices – and through the public display of those choices. I argue that such public displays are always at risk of misinterpretation due to the dominant discursive frameworks through which abstention is given meaning. This paper gives particular attention to the ways in which connotations of taste and distinction are invoked by refusers through their conspicuous displays of non-consumption. This has the effect of framing refusal as a performance of elitism, which may work against observers interpreting conscientious refusal as a persuasive and emulable practice of critique. The implication of this is that refusal is a limited tactic of political engagement where media platforms are concerned.

20. Technology and social inclusion: Rethinking the digital divide

Book

21. Who uses Facebook? An investigation into the relationship between the Big Five, shyness, narcissism, loneliness, and Facebook usage
The unprecedented popularity of the social networking site Facebook raises a number of important questions regarding the impact it has on sociality. However, as Facebook is a very recent social phenomenon, there is a distinct lack of psychological theory relating to its use. While research has begun to identify the types of people who use Facebook, this line of investigation has been limited to student populations. The current study aimed to investigate how personality influences usage or non-usage of Facebook. The sample consisted of 1324 self-selected Australian Internet users (1158 Facebook users and 166 Facebook nonusers), between the ages of 18 and 44. Participants were required to complete an online questionnaire package comprising the Big Five Inventory (BFI), the Narcissistic Personality Inventory – 29-item version (NPI-29), the Revised Cheek and Buss Shyness Scale (RCBS), and the Social and Emotional Loneliness Scale for Adults – Short version (SELSA-S). Facebook users also completed a Facebook usage questionnaire. The results showed that Facebook users tend to be more extraverted and narcissistic, but less conscientious and socially lonely, than nonusers. Furthermore, frequency of Facebook use and preferences for specific features were also shown to vary as a result of certain characteristics, such as neuroticism, loneliness, shyness and narcissism. It is hoped that research in this area continues, and leads to the development of theory regarding the implications and gratifications of Facebook use.

22. From lurkers to posters
http://sielearning.tafensw.edu.au/toolboxes/TAA_ELEARNING_ELECTIVES/taaelearning/toolbox12_10/espace/sources/documents/docs/lurkerstoposters.pdf

no abstract

23. Whose space? Differences among users and non-users of social network sites
Are there systematic differences between people who use social network sites and those who stay away, despite a familiarity with them? Based on data from a survey administered to a diverse group of young adults, this article looks at the predictors of SNS usage, with particular focus on Facebook, MySpace, Xanga, and Friendster. Findings suggest that use of such sites is not randomly distributed across a group of highly wired users. A person’s gender, race and ethnicity, and parental educational background are all associated with use, but in most cases only when the aggregate concept of social network sites is disaggregated by service. Additionally, people with more experience and autonomy of use are more likely to be users of such sites. Unequal participation based on user background suggests that differential adoption of such services may be contributing to digital inequality.

24. The social affordances of the Internet for networked individualism

We review the evidence from a number of surveys in which our NetLab has been involved about the extent to which the Internet is transforming or enhancing community. The studies show that the Internet is used for connectivity locally as well as globally, although the nature of its use varies in different countries. Internet use is adding on to other forms of communication, rather than replacing them. Internet use is reinforcing the pre-existing turn to societies in the developed world that are organized around networked individualism rather than group or local solidarities. The result has important implications for civic involvement.

25. The effects of trust, security and privacy in social networking: A security-based approach to understand the pattern of adoption

Social network services (SNS) focus on building online communities of people who share interests and/or activities, or who are interested in exploring the interests and activities of others. This study examines security, trust, and privacy concerns with regard to social networking Websites among consumers using both reliable scales and measures. It proposes an SNS acceptance model by integrating cognitive as well as affective attitudes as primary influencing factors, which are driven by underlying beliefs, perceived security, perceived privacy, trust, attitude, and intention. Results from a survey of SNS users validate that the proposed theoretical model explains and predicts user acceptance of SNS substantially well. The model shows excellent measurement properties and establishes perceived privacy and perceived security of SNS as distinct constructs. The finding also reveals that perceived security moderates the effect of perceived privacy on trust. Based on the results of this study, practical implications for marketing strategies in SNS markets and theoretical implications are recommended accordingly.

26. It's Complicated: How Romantic Partners Use Facebook

omantic partners face issues of relational development including managing information privacy, tension between individual and relational needs, and accountability to existing friends. Prior work suggests that affordances of social media might highlight and shape these tensions; to explore this, we asked 20 people to reflect daily for two weeks on feelings and decisions around their own and others' Facebook use related to their relationships. Most generally, we find that tensions arise when romantic partners must manage multiple relationships simultaneously because Facebook audiences are so present and so varied. People also engage in subtle negotiation around and appropriation of Facebook's features to accomplish both personal and relational goals. By capturing both why people make these decisions and how Facebook's affordances support them, we expect our findings to generalize to many other social media tools and to inform theorizing about how these tools affect relational development.

27. Non-users also matter: The construction of users and non-users of the Internet
Book

28. Profiling the non-user: Rethinking policy initiatives stimulating ICT acceptance

Business strategies and policies that were successful in increasing internet penetration in the early days may no longer be appropriate. This is most probably so in countries where a bigger proportion of the population is already connected to the internet. As more people are online, it becomes more likely that the remaining fraction of non-users is either hard to convince, under-skilled or simply lacking the financial resources to afford a connection. In view of this, a new policy approach is proposed to increase ICT acceptance. The approach is based on strategies of segmentation and differentiation. This entails that policy initiatives are specifically targeted towards different groups in the population. This article demonstrates that being a non-user can be explained by a combination of access problems, lack of ICT skills or rather negative attitudes towards ICT or by the outweighing effect of one of them. It also provides a framework for setting up new policy measures.

29. The deepening divide: Inequality in the information society

book

30. Beyond the User: Use and Non-use in HCI

For many, an interest in Human-Computer Interaction is equivalent to an interest in usability. However, using computers is only one way of relating to them, and only one topic from which we can learn about interactions between people and technology. Here, we focus on not using computers -- ways not to use them, aspects of not using them, what not using them might mean, and what we might learn by examining non-use as seriously as we examine use.

31. The tone of life on social networking sites

http://www.pewinternet.org/files/old-media/Files/Reports/2012/Pew_Social%20networking%20climate%202.9.12.pdf

No abstract

32. Teens, Kindness and Cruelty on Social Network Sites: How American Teens Navigate the New World of" Digital Citizenship"
Social media use has become so pervasive in the lives of American teens that having a presence on a social network site is almost synonymous with being online. Fully 95% of all teens ages 12-17 are now online and 80% of those online teens are users of social media sites. The authors focused their attention in this research on social network sites because they wanted to understand the types of experiences teens are having there and how they are addressing negative behavior when they see it or experience it. As they navigate challenging social interactions online, who is influencing their sense of what it means to be a good or bad "digital citizen"? How often do they intervene to stand up for others? How often do they join in the mean behavior? Many log on daily to their social network pages and these have become spaces where much of the social activity of teen life is echoed and amplified--in both good and bad ways. In their survey, the authors follow teens' experiences of online cruelty--either personally felt or observed--from incident to resolution. The authors asked them about how they reacted to the experience and how they saw others react. They asked them about whether they have received and where they sought advice--both general advice about online safety and responsibility and specific advice on how to handle a witnessed experience of online cruelty on a social network site. They also probed the environment around teens' online experiences by examining their privacy controls and practices, as well as the level of regulation of their online environment by their parents. The authors further sought insight into more serious experiences that teens have in their lives, including bullying both on- and offline and the exchange of sexually charged digital images. (Contains 2 tables and 57 footnotes.)

33. Stories As Informal Lessons About Security

Non-expert computer users regularly need to make security-relevant decisions; however, these decisions tend not to be particularly good or sophisticated. Nevertheless, their choices are not random. Where does the information come from that these non-experts base their decisions upon? We argue that much of this information comes from stories they hear from other people. We conducted a survey to ask open- and closed- ended questions about security stories people hear from others. We found that most people have learned lessons from stories about security incidents informally from family and friends. These stories impact the way people think about security, and their subsequent behavior when making security-relevant decisions. In addition, many people retell these stories to others, indicating that a single story has the potential to influence multiple people. Understanding how non-experts learn from stories, and what kinds of stories they learn from, can help us figure out new methods for helping these people make better security decisions.

34. Undesigning Technology: Considering the Negation of Design by Design

Motivated by substantive concerns with the limitations and negative effects of technology, this paper inquires into the negation of technology as an explicit and intentional aspect of design research within HCI. Building on theory from areas including philosophy and design theory, this paper articulates a theoretical framework for conceptualizing the intentional negation of technology (i.e., the undesign of technology), ranging from the inhibition of particular uses of technology to the total erasure or foreclosure of technology. The framework is then expanded upon to articulate additional areas of undesigning, including self-inhibition, exclusion, removal, replacement, restoration, and safeguarding. In conclusion a scheme is offered for addressing questions concerning the disciplinary scope of undesign in the context of HCI, along with suggestions for ways that undesigning may be more strongly incorporated within HCI research.

35. Norm evolution and violation on Facebook

This study explores how norms on social network sites evolve over time and how violations of these norms impact individuals’ self-presentational and relationship goals. Employing Expectancy Violations Theory (Burgoon, 1978) as a guiding framework, results from a series of focus groups suggest that both the content of the violation and the users’ relationship to the violator impact how individuals react to negative violations. Specifically, acquaintances who engage in minor negative violations are ignored or hidden, while larger infractions (that could negatively impact the individual) result in deletion of the offending content and – in extreme cases – termination of the Facebook friendship. Negative violations from close friends (that did not impact participants’ goals) resulted in confrontations, while similar violations from acquaintances were often ignored by participants in an effort to ‘keep the peace.’ Furthermore, positive violations were more likely to arise from acquaintances than close friends.

36. Digital Suicide and the Biopolitics of Leaving Facebook

In 2009 leaving Facebook became a trend. Two separate art projects Seppukoo.com and Web 2.0 Suicidemachine presented digital suicide as a method to disconnect oneself from the social networking service. In this article, I approach the problem of leaving Facebook. A critical analysis of leaving Facebook and using digital suicide sites illustrates how life becomes tangled and controlled in the ubiquitous webs of network culture, and moreover how biopolitical models of capitalism are embedded in the structures and practices of exploiting the users of social networks. Theoretically this article is inspired by critical studies of digital culture.

37. Looking at, Looking Up or Keeping Up with People?: Motives and Use of Facebook

This paper investigates the uses of social networking site Facebook, and the gratifications users derive from those uses. In the first study, 137 users generated words or phrases to describe how they used Facebook, and what they enjoyed about their use. These phrases were coded into 46 items which were completed by 241 Facebook users in Study 2. Factor analysis identified seven unique uses and gratifications: social connection, shared identities, content, social investigation, social network surfing and status updating. User demographics, site visit patterns and the use of privacy settings were associated with different uses and gratifications.

38. Un-friend my heart: Facebook, promiscuity, and heartbreak in a neoliberal age

In interviews with Indiana University college students, undergraduates insisted that Facebook could be a threat to their romantic relationships. Some students choose to deactivate their Facebook accounts to preserve their relationships. No other new media was described as harmful. This article explores why Facebook was singled out. I argue that Facebook encourages (but does not require) users to introduce a neoliberal logic to all their intimate relationships, which these particular users believe turns them into selves they do not want to be.

39. Identity Management: Multiple Presentations of Self in Facebook

As the use of social networking websites becomes increasingly common, the types of social relationships managed on these sites are becoming more numerous and diverse. This research seeks to gain an understanding of the issues related to managing different social networks through one system, in particular looking at how users of these systems present themselves when they are using one site to keep in contact with both their past social groups from school and their current social connections in the workplace. To do this, we examined online profile pages and interviewed employees at a large software development company who frequently use the website Facebook, a site primarily used by college students and young graduates transitioning into the work force. The outcome of this initial case study is a framework for understanding how users manage self-presentation while maintaining social relationships in heterogeneous networks.

40. Social network sites: Definition, history, and scholarship

Social network sites (SNSs) are increasingly attracting the attention of academic and industry researchers intrigued by their affordances and reach. This special theme section of the Journal of Computer-Mediated Communication brings together scholarship on these emergent phenomena. In this introductory article, we describe features of SNSs and propose a comprehensive definition. We then present one perspective on the history of such sites, discussing key changes and developments. After briefly summarizing existing scholarship concerning SNSs, we discuss the articles in this special section and conclude with considerations for future research.

41. Adopt, adapt, abandon: Understanding why some young adults start, and then stop, using instant messaging

Instant messaging (IM) has become a popular and important mode of staying in touch for teens and young adults. It allows for easy, frequent and lightweight interaction that contributes to building and sustaining friendships, as well as coordinating social activities. Despite the initial appeal of IM, however, some have found it too distracting and have changed their usage or abandoned it. I interviewed 21 former users of IM about their adoption, usage and eventual abandonment of the technology. Results show that participants were initially attracted to features of IM that enabled them to maximize their use of leisure time via easy and frequent interaction with their friends, but that, in a different usage context, these same features became distracting and annoying. Participants adapted their behavior to avoid these drawbacks, but IM did not support these adaptations effectively. In particular, IM did not allow for control over interruptions, which became more important as their contact lists grew and social time became scarce; and they ultimately abandoned the technology. These results point to a need for understanding use beyond adoption, and a theoretical and practical focus on understanding the adaptation and changing utility that accompany long-term usage of technologies.

42. The social construction of technological systems: New directions in the sociology and history of technology

Book

43. How do older netcitizens compare with their younger counterparts?

The Internet is modifying the lives of people around the world. Although many talk about the democratization of knowledge and information, differences remain among users as older netcitizens are under-represented and less involved. We use national and representative U.S. data, the Current Population Survey, to show age-based differences. We complement our analysis with web-based data, the Georgia Tech World Wide Web User Surveys, to show Internet characteristics and trends by age for netcitizens. Results show that older users compose a lower share of Internet users than that of the total U.S. population; however, once they join the ranks of avid Internet users, older netcitizens are similar to their younger counterparts.

44. The Imagined Audience and Privacy Concern on Facebook: Differences Between Producers and Consumers

Facebook users share information with others by creating posts and specifying who should be able to see each post. Once a user creates a post, those who see it have the ability to copy and re-share the information. But, if the reader has a different understanding of the information in the post than the creator intended, he or she may use the information in ways that are contrary to the intentions of the original creator. This study examined whether post creators (Producers) and readers (Consumers) who are Facebook Friends had similar levels of privacy concern regarding how others might use the information in specific posts, and how their privacy concern about the post varied by whether the imagined audience consisted of Friends, Friends of Friends, or the general Public. The results showed that both Producers and Consumers had similar levels of privacy concern about a post shared with an imagined audience of Friends versus Friends of Friends. However, Consumers believed posts were more private than the Producers themselves did, and showed more privacy concern. This shows that post Consumers care about Producers’ privacy, perceive that they are co-owners of the information, and engage in boundary management with Producers.

45. Who commits virtual identity suicide? Differences in privacy concerns, internet addiction, and personality between Facebook users and quitters

Social networking sites such as Facebook attract millions of users by offering highly interactive social communications. Recently, a counter movement of users has formed, deciding to leave social networks by quitting their accounts (i.e., virtual identity suicide). To investigate whether Facebook quitters (n=310) differ from Facebook users (n=321), we examined privacy concerns, Internet addiction scores, and personality. We found Facebook quitters to be significantly more cautious about their privacy, having higher Internet addiction scores, and being more conscientious than Facebook users. The main self-stated reason for committing virtual identity suicide was privacy concerns (48 percent). Although the adequacy of privacy in online communication has been questioned, privacy is still an important issue in online social communications.

46. Online dating: Americans who are seeking romance use the Internet to help them in their search, but there is still widespread public concern about the safety of online dating

Book

47. Identity construction on Facebook: Digital empowerment in anchored relationships

Early research on online self-presentation mostly focused on identity constructions in anonymous
online environments. Such studies found that individuals tended to engage in role-play games and
anti-normative behaviors in the online world. More recent studies have examined identity performance
in less anonymous online settings such as Internet dating sites and reported different findings.
The present study investigates identity construction on Facebook, a newly emerged nonymous online
environment. Based on content analysis of 63 Facebook accounts, we find that the identities produced
in this nonymous environment differ from those constructed in the anonymous online environments
previously reported. Facebook users predominantly claim their identities implicitly rather
than explicitly; they ‘‘show rather than tell” and stress group and consumer identities over personally
narrated ones. The characteristics of such identities are described and the implications of this finding
are discussed.

48.The Political Economy of Privacy on Facebook
This article provides an analysis of the political economy of privacy and surveillance on Facebook. The concepts of socialist privacy and socialist internet privacy are advanced here. Capital accumulation on Facebook is based on the commodification of users and their data. One can in this context speak, based on Dallas Smythe, of the exploitation of the internet prosumer commodity. Aspects of a socialist internet privacy strategy are outlined and it is shown how they can be applied to social networking sites.

49. All about me: Disclosure in online social networking profiles: The case of FACEBOOK
The present research examined disclosure in online social networking profiles (i.e., FACEBOOK™). Three studies were conducted. First, a scoring tool was developed in order to comprehensively assess the content of the personal profiles. Second, grouping categories (default/standard information, sensitive personal information, and potentially stigmatizing information) were developed to examine information pertinent to identity threat, personal and group threat. Third, a grouping strategy was developed to include all information present in FACEBOOK™, but to organize it in a meaningful way as a function of the content that was presented. Overall, approximately 25% of all possible information that could potentially be disclosed by users was disclosed. Presenting personal information such as gender and age was related to disclosure of other sensitive and highly personal information. Age and relationship status were important factors in determining disclosure. As age increased, the amount of personal information in profiles decreased. Those seeking a relationship were at greatest risk of threat, and disclosed the greatest amount of highly sensitive and potentially stigmatizing information. These implications of these findings with respect to social and legal threats, and potential means for identifying users placing themselves at greatest risk, are discussed.


50. Why study technology non-use?

This special issue provides an opportunity to rethink how we approach, study, and conceptualize human relationships with, and through, technology. The authors in this collection take a multiplicity of approaches on diverse topics to develop a rigorous theoretical understanding for non-use, setting crucial groundwork for future research.

51. On the importance and implications of studying technology non-use

%Unable to obtain the abstract via VPN. Will try again after reaching Bloomington

52. Missing photos, suffering withdrawal, or finding freedom? How experiences of social media non-use influence the likelihood of reversion

This article examines social media reversion, when a user intentionally ceases using a social media site but then later resumes use of the site. We analyze a convenience sample of survey data from people who volunteered to stay off Facebook for 99 days but, in some cases, returned before that time. We conduct three separate analyses to triangulate on the phenomenon of reversion: simple quantitative predictors of reversion, factor analysis of adjectives used by respondents to describe their experiences of not using Facebook, and statistical topic analysis of free-text responses. Significant factors predicting either increased or decreased likelihood of reversion include, among others, prior use of Facebook, experiences associated with perceived addiction, issues of social boundary negotiation such as privacy and surveillance, use of other social media, and friends’ reactions to non-use. These findings contribute to the growing literature on technology non-use by demonstrating how social media users negotiate, both with each other and with themselves, among types and degrees of use and non-use.

53. Comparing grounded theory and topic modeling: Extreme divergence or unlikely convergence?

Researchers in information science and related areas have developed various methods for analyzing textual data, such as survey responses. This article describes the application of analysis methods from two distinct fields, one method from interpretive social science and one method from statistical machine learning, to the same survey data. The results show that the two analyses produce some similar and some complementary insights about the phenomenon of interest, in this case, nonuse of social media. We compare both the processes of conducting these analyses and the results they produce to derive insights about each method's unique advantages and drawbacks, as well as the broader roles that these methods play in the respective fields where they are often used. These insights allow us to make more informed decisions about the tradeoffs in choosing different methods for analyzing textual data. Furthermore, this comparison suggests ways that such methods might be combined in novel and compelling ways.

54. When the Implication is Not to Design (Technology)

As HCI is applied in increasingly diverse contexts, it is important to consider situations in which computational or information technologies may be less appropriate. This paper presents a series of questions that can help researchers, designers, and practitioners articulate a technology's appropriateness or inappropriateness. Use of these questions is demonstrated via examples from the literature. The paper concludes with specific arguments for improving the conduct of HCI. This paper provides a means for understanding and articulating the limits of HCI technologies, an important but heretofore under-explored contribution to the field.

55. When Subjects Interpret the Data: Social Media Non-use as a Case for Adapting the Delphi Method to CSCW

This paper describes the use of the Delphi method as a means of incorporating study participants into the processes of data analysis and interpretation. As a case study, it focuses on perceptions about use and non-use of the social media site Facebook. The work presented here involves three phases. First, a large survey included both a demographically representative sample and a convenience sample. Second, a smaller follow-up survey presented results from that survey back to survey respondents. Third, a series of qualitative member checking interviews with additional survey respondents served to validate the findings of the follow-up survey. This paper demonstrates the utility of Delphi by highlighting the ways that it enables us to synthesize across these three study phases, advancing understanding of perceptions about social media use and non-use. The paper concludes by discussing the broader applicability of the Delphi method across CSCW research.

56. Privacy Personas: Clustering Users via Attitudes and Behaviors Toward Security Practices

A primary goal of research in usable security and privacy is to understand the differences and similarities between users. While past researchers have clustered users into different groups, past categories of users have proven to be poor predictors of end-user behaviors. In this paper, we perform an alternative clustering of users based on their behaviors. Through the analysis of data from surveys and interviews of participants, we identify five user clusters that emerge from end-user behaviors-Fundamentalists, Lazy Experts, Technicians, Amateurs and the Marginally Concerned. We examine the stability of our clusters through a survey-based study of an alternative sample, showing that clustering remains consistent. We conduct a small-scale design study to demonstrate the utility of our clusters in design. Finally, we argue that our clusters complement past work in understanding privacy choices, and that our categorization technique can aid in the design of new computer security technologies.

57. Can an algorithm be wrong? Twitter Trends, the specter of censorship, and our faith in the algorithms around us

no abstract

58. my data just goes everywhere:” user mental models of the internet and implications for privacy and security

Many people use the Internet every day yet know little about how
it really works. Prior literature diverges on how people’s Internet
knowledge affects their privacy and security decisions. We
undertook a qualitative study to understand what people do and do
not know about the Internet and how that knowledge affects their
responses to privacy and security risks. Lay people, as compared
to those with computer science or related backgrounds, had
simpler mental models that omitted Internet levels, organizations,
and entities. People with more articulated technical models
perceived more privacy threats, possibly driven by their more
accurate understanding of where specific risks could occur in the
network. Despite these differences, we did not find a direct
relationship between people’s technical background and the
actions they took to control their privacy or increase their security
online. Consistent with other work on user knowledge and
experience, our study suggests a greater emphasis on policies and
systems that protect privacy and security without relying too
much on users’ security practices

59. Understanding social network site users’ privacy tool use

Every day hundreds of millions of people log into social network sites and deposit terabytes of data as they share status updates, photographs, and more. This article explores how background factors, motivations, and social network site experiences relate to people’s use of social network site technology to protect their privacy. The findings indicate that during technology-mediated communication on social network sites, not only do traditional privacy factors relate to the technological boundaries people enact, but people’s experiences with the mediating technology itself do, too. The results also identify privacy inequalities, in which certain groups are more likely to take advantage of the technology to protect their privacy—suggesting that some individuals’ information and reputations may be more at risk than others’.

60. It's Creepy, But It Doesn't Bother Me

Undergraduates interviewed about privacy concerns related to online data collection made apparently contradictory statements. The same issue could evoke concern or not in the span of an interview, sometimes even a single sentence. Drawing on dual-process theories from psychology, we argue that some of the apparent contradictions can be resolved if privacy concern is divided into two components we call intuitive concern, a "gut feeling," and considered concern, produced by a weighing of risks and benefits. Consistent with previous explanations of the so-called privacy paradox, we argue that people may express high considered concern when prompted, but in practice act on low intuitive concern without a considered assessment. We also suggest a new explanation: a considered assessment can override an intuitive assessment of high concern without eliminating it. Here, people may choose rationally to accept a privacy risk but still express intuitive concern when prompted.

61. Americans’ privacy strategies post-Snowden

article. pew research. we might discard the article

62. Giving Up Twitter for Lent: How and Why We Take Breaks from Social Media

Social media use is widespread, but many people worry about overuse. This paper explores how and why people take breaks from social media. Using a mixed methods approach, we pair data from users who tweeted about giving up Twitter for Lent with an interview study of social media users. We find that 64% of users who proclaim that they are giving up Twitter for Lent successfully do so. Among those who fail, 31% acknowledge their failure; the other 69% simply return. We observe hedging patterns (e.g. "I thought about giving up Twitter for Lent but"?) that surfaced uncertainty about social media behavior. Interview participants were concerned about the tradeoffs of spending time on social media versus doing other things and of spending time on social media rather than in "real life." We discuss gaps in related theory that might help reduce users' anxieties and open design problems related to designing systems and services that can help users manage their own social media use.

63. Whose Internet is it anyway? Exploring adults’(non) use of the Internet in everyday life

It is acknowledged that communication researchers need to develop more sophisticated and nuanced accounts of the social and individual dynamics of the internet in everyday life. Based on a household survey of 1001 adults with 100 in-depth follow-up interviews, the present article explores people’s (non)use of the internet by asking: (1) who is (and who is not) using the internet in everyday life; (2) for what purposes people are using the internet and how are they developing their own constructions of the internet; and (3) how these understandings and uses of the internet are shaped by existing socioeconomic factors and circumstances. From this basis the article goes on to identify the key issues underlying adults’ (non)use of the internet in terms of interest, relevance, mediation of significant others and the role of household dynamics. It also considers, from this basis, how non-users may be encouraged to make use of the internet.

64. Apart from technology: understanding people’s non-use of information and communication technologies in everyday life

Despite the high-profile nature of the current ‘digital divide’ debate, academic understanding of who is making little or no use of information and communication technologies (ICTs) remains weak. Indeed much of the discussion surrounding the digital divide has concentrated on the characteristics of those individuals who are using ICTs or, at best, simply pathologised the ‘have nots’ in terms of individual deficits. Yet developing a systematic and objective understanding of individuals’ non-use of new technologies constitutes a major challenge for those seeking to map and understand the social realities of the ‘information age’. The present paper, therefore, aims to develop a deeper conceptual understanding of people’s non-use of new technologies: firstly, by considering established discourses of why individuals may be excluded or peripheral to ICT use; and then, via a critique of these positions, proposing an alternative framework of why people may not use ICT in their day-to-day lives based around individuals’ ‘reading’ of technology.

65. It's All About Networking! Empirical Investigation of Social Capital Formation on Social Network Sites

As Social Network Sites (SNS) permeate our daily routines, the question whether participation results in value for SNS users becomes particularly acute. This study adopts a 'participation-source-outcome' perspective to explore how distinct uses of SNS generate various types of social capital benefits. Building on existing research, extensive qualitative findings and an empirical study with 253 Facebook users, we uncover the process of social capital formation on SNS. We find that even though active communication is an important prerequisite, it is the diversified network structure and the increased social connectedness that are responsible for the attainment of the four benefits of social capital on SNS: emotional support, networking value, horizon broadening and offline participation. Moreover, we propose and validate scales to measure social capital benefits in the novel context of SNS.

66. Facebook and online privacy: Attitudes, behaviors, and unintended consequences

This article investigates Facebook users' awareness of privacy issues and perceived benefits and risks of utilizing Facebook. Research found that Facebook is deeply integrated in users' daily lives through specific routines and rituals. Users claimed to understand privacy issues, yet reported uploading large amounts of personal information. Risks to privacy invasion were ascribed more to others than to the self. However, users reporting privacy invasion were more likely to change privacy settings than those merely hearing about others' privacy invasions. Results suggest that this lax attitude may be based on a combination of high gratification, usage patterns, and a psychological mechanism similar to third‐person effect. Safer use of social network services would thus require changes in user attitude.

67. Analyzing Facebook Privacy Settings: User Expectations vs. Reality

The sharing of personal data has emerged as a popular activity over online social networking sites like Facebook. As a result, the issue of online social network privacy has received significant attention in both the research literature and the mainstream media. Our overarching goal is to improve defaults and provide better tools for managing privacy, but we are limited by the fact that the full extent of the privacy problem remains unknown; there is little quantification of the incidence of incorrect privacy settings or the difficulty users face when managing their privacy.

In this paper, we focus on measuring the disparity between the desired and actual privacy settings, quantifying the magnitude of the problem of managing privacy. We deploy a survey, implemented as a Facebook application, to 200 Facebook users recruited via Amazon Mechanical Turk. We find that 36% of content remains shared with the default privacy settings. We also find that, overall, privacy settings match users' expectations only 37% of the time, and when incorrect, almost always expose content to more users than expected. Finally, we explore how our results have potential to assist users in selecting appropriate privacy settings by examining the user-created friend lists. We find that these have significant correlation with the social network, suggesting that information from the social network may be helpful in implementing new tools for managing privacy.

68. Facebook privacy settings: Who cares?


With over 500 million users, the decisions that Facebook makes about its privacy settings have the potential to influence many people. While its changes in this domain have often prompted privacy advocates and news media to critique the company, Facebook has continued to attract more users to its service. This raises a question about whether or not Facebook's changes in privacy approaches matter and, if so, to whom. This paper examines the attitudes and practices of a cohort of 18- and 19-year-olds surveyed in 2009 and again in 2010 about Facebook's privacy settings. Our results challenge widespread assumptions that youth do not care about and are not engaged with navigating privacy. We find that, while not universal, modifications to privacy settings have increased during a year in which Facebook's approach to privacy was hotly contested. We also find that both frequency and type of Facebook use as well as Internet skill are correlated with making modifications to privacy settings. In contrast, we observe few gender differences in how young adults approach their Facebook privacy settings, which is notable given that gender differences exist in so many other domains online. We discuss the possible reasons for our findings and their implications.

69. Refusing, Limiting, Departing: Why We Should Study Technology Non-use

In contrast to most research in HCI, this workshop focuses
on non-use, that is, situations where people do
not use computing technology. Using a reflexive preworkshop
activity and discussion-oriented sessions, we
will consider the theories, methods, foundational texts,
and central research questions in the study of non-use.
In addition to a special issue proposal, we expect the research
thread brought to the fore in this workshop will
speak to foundational questions of use and the user in
HCI.


\end{comment}
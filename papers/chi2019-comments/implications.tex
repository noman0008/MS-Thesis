% !TEX root=nonusecomments.tex
\section{Implications}
\label{sec:implications}
The implications of the findings are manifold: from the perspective of researchers, users, service providers, and system designers~\cite{baumer2015importance}. A common psychology in practice is people go with the trend or they tend to follow the crowd ~\cite{gilbert2009blogs}. So from a user's point of view, when other blog readers of Slashdot see such negative emotions associated with Facebook (or any other social media in general), they tend to get biased as well. This is mainly because since the Slashdot and Schneier's blog users are presumably technology geeks, their comments or reviews carry higher weight than usual. These people are early adopters and their knowledge stems from deeper understanding of technology. So if they are expressing \emph{NS} and promoting \textit{Non-use}, decision making of non-experts in terms of social media usage is negatively affected as well. Our \textit{RQ1} results shed light on this. 


Our \textit{RQ2} results have implications on the way future systems could be designed and existing systems could be evaluated. For example, an independent researcher or system designer wants to evaluate any existing blog or forum for user satisfaction level. With help of the techniques adopted in answering \textit{RQ1} and \textit{RQ2} or a similar but modified approach, a sentiment classifier can be built and user sentiment for that particular blog/forum can be assessed. This way product business related blog/forums could be benefited too because such blogs contain high volume unprompted reviews from users (e.g., Yelp, Amazon) and their target audience are the buyers. Our approach will enable them to understand user behavior or sentiment and adjust their business model accordingly. This knowledge can also be applied for developing a new media for communication too since traditional usability study of newly developed system is time consuming. We believe the most important implication for our \textit{RQ2} results is practical. For instance, a Facebook system analyst can go through the finding of this work to get a top level idea of what is affecting user engagement and ways to fix them. Leveraging the findings of the 10 influential factors and sentiment associated with them, it can be deduced that Facebook needs to improve their system architecture and enhance privacy and security measures. The findings offer implications for system designers and HCI researchers too. Our findings suggest that, for both dataset users are mostly concerned with privacy and security issues while user experience and personal preference play a secondary role. 

It has to be understood that technology use is more a cultural phenomenon~\cite{satchell2009beyond} so the study of \emph{Non-use} and use has to be considered from that perspective. The findings do not necessarily point to a radical design change or complete disposal rather it suggests naturalistic responses on what are design flaws and what changes can incorporate more users. Many users suggested how staying away from social media for a while helped them to better balance their social life. So Facebook designers can consider \emph{Non-use} positively and possibly facilitate it by allowing the users to take temporary breaks, strategies to balance user interaction with social media~\cite{schoenebeck2014giving} and enabling more self-inhibiting options without `undesigning' it~\cite{pierce2012undesigning}. As Baumer and Silverman~\cite{baumer2011implication} argued, designers should consider the implication when not to design~\cite{dourish2006implications} but this does not promote rejection of technology. One Slashdotter addressed this issue in his comment:
\begin{quote}
         \textit{You never really leave Facebook. I thought I had deleted my account but signed back in a month later and it was still there. All my friends still present. Even the pictures and comments I had deleted individually were still there.}
    \end{quote}
Exact similar sentiment regarding Facebook account deactivation was found by another user of Schneier on Security: \textit{``My question is if I delete my Facebook is it ever really gone? I would love to get rid of it but my friends would think I'm no longer friends with them and take it to heart which I think is stupid. I've heard of stories of deleting a Facebook then coming back years later and everything is just how you left it.''} It is evident that Facebook currently disregard the ongoing movement in terms of \emph{Non-use}. Our findings have critical implications for UI designers of Facebook in this regard. Incorporating ideas from non-users guarantee an engaged user-base and can bring positive sociotechnical changes~\cite{baumer2013limiting}. On the other hand, if they do not facilitate users who willingly decide to withdraw or recess, the looming \emph{NS} of Facebook users can be menacing for them. Today the non-users may not be an overwhelming majority but their perspective should be valued and dealt with~\cite{wyatt2003non, satchell2009beyond}, especially if they are tech-savvy users. A more deeper analysis will reveal other system pitfalls and ways to overcome them. This is significant for users and developers alike because it will enable them to create an all inclusive environment that is devoid of the factors discovered in our findings.
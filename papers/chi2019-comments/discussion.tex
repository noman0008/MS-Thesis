% !TEX root=nonusecomments.tex
\section{Discussion}
\label{sec:discussion}
\begin{quote}
\textit{\textbf{RQ1}: Do expert users possess negative sentiment towards technology and are actually rejecting it (Facebook in our case)?} (By rejecting we mean reduced or passive usage, dropping out, and resistance towards Facebook.)
\end{quote}

Our findings from the initial level of coding (binary) is very intriguing. From a complete random sample of 3000 (2000 \& 1000 from Slashdot and Schneier's blog respectively) Facebook related comments, we observed 51.1\% \& 32\%, respectively, of total comments those were indicative of some form of Facebook \textit{NS} and \emph{Non-use}. This shows the propensity for resisting usage, lurking or abandoning a large scale well-designed system like Facebook by security experts and tech-savvy users. Considering all the Facebook related comments of these two blogs belong to either one of the class \textit{NS} or \textit{not related to NS}, the probability of getting a Facebook hater or non-user is nearly 44.73\% in our data (which is very high considering the wide range of other topics belonging to the class \textit{not related to NS}). The discrepancy between the fact that Facebook's user base is growing\footnote{\url{https://newsroom.fb.com/company-info/}}~\cite{lenhart2010social} while we get such a high percentage of expert users who are dissatisfied with Facebook, provides strong evidence to a positive answer to our \emph{RQ1}. This also require in-depth analysis of the factors that play pivotal role in such \textit{NS}. 

\begin{quote}
\textit{\textbf{RQ2}: What factors underlie social media \textit{NS}, non-usage and non-adoption by techies?}
\end{quote}

The second level thematic coding and further analysis attempt to answer \emph{RQ2}. The top 3 out of 10 common themes or primary influential factors for Facebook \textit{NS} which emerged from our findings are: \textit{privacy and security concerns}, \textit{user experience}, and \textit{personal preference}. This implies that users are dissatisfied with several aspects of Facebook system architecture and other features of Facebook in general. This includes but not limited to Facebook ads, Facebook interface, autoplay video feature of Facebook, fake news spreading, insufficient security measure to user content, Facebook policy, Facebook terms and conditions, the way Facebook uses user data to their benefit etc. The myriads of \textit{user experience} issues implies that Facebook is not wary of expert user's desired environment and they are often criticized for doing too much experiments with their UI. A subset of these experiences were also observed in prior works~\cite{lampe2013users, baumer2013limiting, baker2011their}, but this is in bigger scale for our data thus require more nuanced understanding in the future. The other major concern that the users have is lack of privacy and security measures from Facebook point of view. They don't believe Facebook is a reliable storage of their content. Commenters have a common complain regarding data breach and the fact that Facebook is spying on their activity. According to them, Facebook treats the user content as their property and they sell it for their sake. Another issue that was often addressed is Facebook access to the user data, for example, contact list of their phone and they frequently data mine user activity based on search history. We found that current users are particularly concerned about privacy violations by Facebook compared to early days. Does this affect the way people interact with Facebook and how much does it hurt FB itself? Well, the latest Cambridge Analytica debacle was a major setback for Facebook and it will haunt them if remedial measures are not taken. But why tech-savvy users are so prone to Facebook's privacy violation and security loopholes? A possible explanation is: not necessarily their background knowledge is associated with their secure behavior online~\cite{kang2015my}. 

Another theme that evoked inevitably with this was \textit{advertisement} (118 total comments). Most of the \emph{NS} comments of this category revealed that while some users find the Facebook's business model of showing ``targeted ads'' useful, they absolutely reprimanded Facebook for the selling of their personal information to third party or government funded surveillance operations. \textit{Personal preference} is the next factor and it accounts for almost 17.95\% of the combined dataset. It turns out that often times people say, \textit{``I don't care...''}, things like \textit{``It is my choice...''}, \textit{``I hate Facebook because it is wasting my time...''} or they do not express any specific reason rather they just express their disgust against Facebook and advocates for \emph{Non-use}. So, they are not blaming anything else rather their attitude indicates that they are deliberately choosing not to use Facebook for some personal reason. Another thing that came up in this regard was: people don't actually think Facebook friends are real friends rather they underscore the importance of getting out and socializing in real life (Lampe et al.~\cite{lampe2013users} described this as \textit{channel effect}). This category of comments also include users castigating Facebook (e.g., by using slang or swearing terms, such as ``fuck'', ``shit'', ``creepy'', ``sucks'' etc.) but did not really point out why.  

Some Facebook users are not particularly happy with the \textit{uninteresting/unverified content} or in particular the audience activity; such as fake friends, distasteful confrontation among members of any Facebook group, presence of parents in friend list (also explained as \textit{disenfranchisement}~\cite{satchell2009beyond}), hoax/propaganda, deliberate misinformation, trolling, cyberbullying, overuse of abusive language, religious bigotry etc. Another frequently discussed important point was fake news, which is particularly annoying to users who are reliant on Facebook for information around the globe. This category also explains the immediate next category \textit{psychological impact}. We believe the over exposure to such non-relevant contents and ostentatious display of personal details of friends on Facebook can cause damaging psychological impact. The most cited ones are depression, addiction to social media etc., which is consistent with the findings of prior work~\cite{moreno2011feeling, christakis_2017}. One possible explanation of this is to maintain a positive impression management on Facebook~\cite{rosenberg2011online} or that Facebook users are primarily extraverted and narcissistic~\cite{ryan2011uses}. Another form of \textit{Non-use}, such as lurking, was observed in our data when users expressed why they are compelled to use Facebook against their will because they care about their Facebook friends and they simply find no other option but to use it since relatives/friends/employers take it for granted that everyone has to have an Facebook account. Baumer et al.~\cite{baumer2013limiting} described such an event as \textit{lagging resistance}. This phenomena can also be explained by influence of external factors in social media use or \emph{Non-use} when users are inadvertently forced to stay on Facebook for reasons like just to keep in touch with others, get information about something which otherwise is not possible. 

Political and government/external interception on Facebook use were also discussed by some users (31 total comments). In general, users do not mind seeing political content in their Facebook feed. However, the comments indicate \emph{NS} when politics/political contents are used as a medium to create anarchy and unrest. The latest US election and they way social media was used to understand voters sentiment explains such comments. Some of the users found some non-Facebook media use is more user friendly or a better mode of communication compared to Facebook (e.g., Slack, Instagram, Snapchat, Reddit etc.). Such comments indicated that it is not necessarily has to be another social media rather they mentioned mere emails can be a good enough alternative to make life easier. 

Among all these categories the common theme expressed by dissatisfied users is the fact that the cost of adopting the technology is simply outweighing the perceived benefits. These findings are congruous with Technology Acceptance Model~\cite{venkatesh2003user, davis1989perceived}. Apparently the users are risking their privacy, time, relationships and in return they are not getting enough return on investment. A more nuanced analysis on the risk factors and the threshold to which users agree to take the risk without compromising privacy and other factors could be a good future work. 


\paragraph{Comparison between Slashdot and Schneier's blog} 

Besides getting additional insight and more data, one motivation to use two different data sources was to compare between the \emph{NS} findings of two separate set of users. It turns out the major source of concern is as expected: \textit{privacy and security concerns}. It is not surprising to see though the percentage of comments for Schneier's blog (66.25\%) is higher than that of Slashdot (42.27\%) since the former one is a predominantly security blog and most discussed topics are state of the art technology security issues. The interesting part is the second prominent theme. While it is \textit{user experience} (33.36\%) for Slashdot, Schneier's blog users expressed \textit{personal preference} (29.37\%). It explains that Schneier's blog users are not very concerned about \textit{user experience} (12.81\%) rather they are more harsh towards Facebook's lack of respect to user privacy. Most of the privacy \& security related concerns were common in both dataset, however, one interesting observation in Schneier's blog comments was that the dissatisfied users think Facebook is a NSA/CIA company. It should be noted that for Slashdot, \textit{personal preference} is the next most cited theme (14.38\%) because we observed most Slashdot comments are shorter in length (average length ~358 characters) compared to Schneier's comments (average length ~984 characters), they are typically direct in nature and sometimes lacks contextual information. So the 3 most prominent themes are same for both dataset but in slightly different order (probably because difference in perspective) which confirms our assumption that they agree to a certain extent. Couple of other significant observation that can be made from Table~\ref{table:coding category} are: first, \textit{psychological impact} is not well represented compared to other themes for both dataset and second, commenters of Schneier on Security are less troubled by \textit{politics/political content} (just 1 comment) compared to Slashdotters (30 comments). The explanation for this and nuances between the other themes for these 2 different sets of audiences require demographic information and further data which is left as a future work.


\paragraph{A closer look at Explicit Non-use only comments}

The assumption we made for analyzing the Facebook \emph{NS} comments was having \emph{NS} towards Facebook might provoke \emph{Non-use} of some form and not always people express everything explicitly in social media unless they are interviewed or surveyed with directed questioning. This also justifies our use of naturalistic user-generated content. Later, in the second level of thematic coding 311 out of 1022 \emph{NS} comments (30.43\%) of Slashdot and 161 out of 320 \emph{NS} comments (50.31\%) of Schneier's blog was marked as \emph{explicit Non-use} where the users explicitly mentioned that they are engaged in reduced or passive usage, dropping out, and resistance. The significant difference in percentage in our sample dataset could be either due to imbalance size of sample for each blog or commenters of Schneier's blog are comparatively more inclined to \emph{Non-use} in general. This can't be verified unless we get more data to analyze. A closer look at the distribution of the themes for \emph{explicit Non-use} only comments (Table~\ref{table:coding category}) reveal similar ordering (top 3) for both dataset as described in previous paragraph. As insinuated by Stieger et al.~\cite{stieger2013commits}, \textit{privacy and security concerns} is the prime reason for committing ``virtual identity suicide'' by Facebook non-users. Majority of the themes are coherent with the prior studies~\cite{lampe2013users, baumer2013limiting, baker2011their} on this issue, however, \textit{politics/political content}, Facebook \textit{user experience} or issues with its system architecture and \textit{fake accounts/bots} are three distinct findings in our data. We do not claim this is a novel discovery rather we want to attribute this to our choice of naturalistic data by tech-savvy audience. Also our assumption is that in recent years non-users are experiencing these problems more than earlier. A better understanding can be obtained by temporal analysis and interaction of these variables in different levels of \textit{Non-use}.

Our observation for both dataset is consistent with the \textit{lagging adoption} concept~\cite{satchell2009beyond} and \textit{diffusion theory}~\cite{selwyn2003apart} where the expert tech-savvy users are on the left of technology adoption bell curve and non-users on the other side. However, those who willingly did not join Facebook have strong reasoning for not doing so and therefore can not be explained by the notion of ``laggards'' proposed by them. Similarly other forms of \emph{Non-use} such as \textit{disenchantment}, \textit{disinterest}~\cite{satchell2009beyond} can be explained by \textit{uninteresting/unverified content} and \textit{personal preference} respectively. Although suffering withdrawal symptoms~\cite{baumer2015missing} were acknowledged by non-users who eventually stopped using Facebook we did not find enough evidence of reversion. For example, one commenter wrote: \textit{``I deleted my Facebook account years ago and I don't miss it at all. It's a great tool for companies to exploit consumers though, that's probably its main purpose these days.''}
 

\begin{comment}


%A closer look at the comments reveals predominant words like "social", "privacy", "fake", "ad", "data", "news", "personal", "security", "stopped", "deleted" which are associated with comment phrases like "social media", "privacy and security concerns", "fake news", "selling personal data", "stopped using", "deleted account" etc. It also contains sentiment terms, such as "shit", "fuck", "hate" etc. that reflect user aversion. The key terms in the bigram comments appear to be "fake news", "stop using", "news feed", "personal information", "privacy setting", "delete account". Most visible key phrases of trigram comments are: "i don't use", "i don't have", "don't use Facebook", "don't have a", "i don't care", "stop using Facebook". These visualizations are informative, for example, terms like "don't use Facebook" appeared 17 times which implies at least 17 times in a set of 516 comments users are talking about not using Facebook. All these gives us a top level idea of if people are rejecting Facebook and why. 

%The interpretation of topics in the comments also supports our findings so far. Top words associated with each topic assumes the corpus is a mixture of these topics. For example "hate", "privacy", "news", "ads", "block", "data", "private", "privacy", "information", "stopped", "left", "dropped", "social", "personal", "deleted", "fake", "stop", "money", "problem" etc. The repetition of similar words indicates that the manual coding criteria adopted by the coders and future automated analysis tend to converge. This is exciting because that answers our \textit{RQ2} and it has practical implications for large scale analysis that will be discussed in Implications section.

\end{comment}

% The results of logistic regression classifier is reported in Table ~\ref{tab:table5}. The interpretation of results of VADER tool does not give a clear idea of how many negative comments belong to each theme because the definition of positiveness and negativeness is dependent upon a normalized score which can be interpreted in different ways. The machine learning approach, although not trained with domain specific data, gives a rough idea of what is the percentage of comments those are more negative per theme. From the results we can see the highest negative comments are from \textit{Non-Facebook Media Use}, \textit{Personal Preferences} and \textit{Audience} respectively. We assume this is because our negative training examples had more similarity with comments from these themes. Incorporating more data for training and using more  sophisticated parameters like Random Forests might be more meaningful which we leave as future work.

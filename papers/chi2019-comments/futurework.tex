% !TEX root=nonusecomments.tex
\section{Future Work}
\label{sec:futurework}
Our findings engender interesting potential research questions: do we get similar user sentiments towards other non-Facebook privacy sensitive media? Does incorporating more data and application of machine learning based text classification approach guarantee successful prediction? For our future work we want to explore other sources where users discuss their Facebook user experience to get a comprehensive understanding. In particular, this will be interesting to analyze if the underlying factors found in this study from expert users are consistent with other data sources consisting of general audience. This however, is not feasible to answer in a single paper, since everyday more and more such platforms are emerging and also analysis of a particular site provides answers to problems of specific audience. Design and implementation of a text classifier, which will significantly reduce the human effort and automate the process, is also left as future work. Unsupervised machine learning based approaches, such as clustering, topic modeling, opinion mining can be applied to detect inherent motives automatically. Also, we would like to extend the knowledge gathered from this study to address more generic agenda, such as overall social media (not only Facebook) sentiment mining. We want to use more hand annotated data for future analysis as it will not only increase the accuracy but also help getting our domain specific sentiment.
% !TEX root=nonuseinterviews.tex
\section{Findings and Discussion}
\label{sec:findings}
% organize by 4 sections- individual, system,... quotes..
%This section presents our findings. It is organized in the way the coding were done to emphasize more on reasons behind the online user behaviour. At first, we provide the demographic description of the participants, next in the following sections, we provide theme wise descriptive summary of the reasons mentioned by the participants in-active usage, no usage, or reduced usage of Facebook.
%\subsection{Participant Profiles}
%We received 248 responses during the pre-screening survey: 224 respondents had Facebook accounts, 17 didn't have a Facebook account, and 78 reduced their usage from last year. We then selected N=22 interview participants depending on whether they do not have any Facebook account or have reduced their usage since last year. We intended to have participants from different age range, gender, and occupation/education to obtain a diverse response. Table~\ref{table:demo} provides the demographics of the participant pool. Out of the given responses we wanted to obtain participants from varied age and occupation range. We also wanted a mix of female versus male in our participant pool- 10 Females, 11 Males, and 1 who Did not want to specify their gender; the mean age was 25.41( We targeted the younger crowd for our study). 

\subsection{Producer, Consumer, and Procsumer}
For our analysis, we asked the participants to classify themselves as ``producer'' or ``consumers'' of information, depending on an individual's activities over Facebook. Three of our participants, P6, P9, and P10 preferred to term themselves as ``procsumers'' who both produced and consumed similar level of content over Facebook, although their usage decreased over the last year. It is not terribly uncommon for the average user on Facebook to consider themselves as a consumer, since in the eyes of most Facebook users, the level of posting needed to be considered a producer is quite large, P1, P2, P5, P7, P8, P11, P13, P14, P16, P17, P18, P19, P20 all termed themselves as consumers. One of our participants, P8 noted that the posting habits were typically sharing or "reposting" news articles and statuses posted by others. The participants mentioned sharing articles and songs on Facebook, also but that the posting of their own photos curtailed after the revelation that the photos uploaded on Facebook became the property of Facebook. In the next section, we explore the activities users perform on social media, thereafter exploring more on why users have modified their usage of Facebook often curtailing the usage or abandoning the platform all together.

\subsection{Activity}
We explored the type of activities users performed to "fact check" their own description of their use of content on Facebook. In order to have more accurate results, we have divided the interviewees on the basis of their usage pattern.

\subsubsection{Frequent Users}
We tried to select participants who indicated they have reduced their usage from last year. However, we observed that even those who wished to reduce their usage found great obstacles in doing so. Most frequent users indicated that they wished to reduce their social media usage but it had become a kind of addiction or habit. On the other hand, those who use most of their time on social media consuming information are generally annoyed with those who "over-post" or are constantly producing content. They feel as though Facebook and other social media platforms should be used for sharing pictures of vacations or announcing big life changing events.
\begin{quote}
P11:\textit{``It has to be big events, went to things that are like a milestone of my life, freely worth to inform other people about. Other than that, if I am going to have breakfast with someone I am not going to post about it."}
\end{quote}

\subsubsection{Facebook as a Broadcasting Medium}
With the current acceleration in the speed at which we receive information, many Facebook users appreciate that they can receive news updates quickly from the site. However Participants noticed websites like Facebook are using algorithms that use your search history and locations visited to show you news articles and ads that you may be interested in. This is not appealing to some but not many have decided to completely give up social media because of this buy have curtailed their use. Other users enjoy it, and like that Facebook is showing them things they care about and have an interest in.
\begin{quote}
P5:\textit{``It's a quick way to spread information. It is like giving people an update, it's been a while. An update that's not really personal, it's like am still alive, everything is moving."}
\end{quote}
%\subsubsection{Facebook Integration into Everyday Activities}
%Most businesses, whether they are brick and mortar stores or entirely based online, are attempting to have a connection with social media websites. Some do so through ads, they create ads based on their target audience and send it to Facebook along with a payment asking them to show the ad to people whom they wish to visit their store or website. Another strategy currently used by businesses is integration straight into Facebook. Users can log in to other websites using their Facebook account, so the website is now able to see what the person might be interested in, showing them products they are more likely to buy. This is advertised as creating convenience for users and businesses but it also effects the privacy of users greatly. Some users choose to keep their accounts separate at an attempt for anonymity.


\subsection{Audience Influence}
There is quite a separation between the types of social media users and the audience they bring onto their pages. Users who generally fall into the producers category typically have their profiles open, allowing anyone who would wish to see their profile to see it. Those who fall more into the consumers side of social media are more concerned with their privacy and image, keeping their profile private or not posting in order to hide their activities from family, friends, or potential employers.

\subsubsection{Self-reflection}
Most users eventually choose to "disconnect" or "unplug" from social media, whether they are consumers or producers regardless. Most stated that they are doing so because they are tired, and overwhelmed with the constant flow of information from their connections and also in constant flux of impression management in front of their audience. They missed the real offline connection with others, the constant fear of what the audiences are interpreting is scary. Because having a social media profile is so prevalent in today's world, people even forget to share something with others in person, assuming they have seen it on Facebook. 
\begin{quote}
P14:\textit{``I feel like most of the time when people post photos it's like, good photos, that portray them in like, a positive light and not a negative light, so they probably, well I do have a good life I feel like so they probably think I have a good life from the photos I post and the statuses are always like I said like substantial material, so, they probably feel like that I think, or they probably can see that I only like, feel the need to post like, important things."}
\end{quote} 

\subsubsection{Older People's Attempt to Join}
Looking at the history of Facebook and other social media sites, it typically takes longer for older people to adopt them. In the beginning of Facebook it was generally young people, mid-twenties and younger, that held accounts. As time has grown, so has Facebook's demographic. More and more older people are adopting it, and are active users while younger people have gone on to Snapchat, Instagram, or Twitter. As time goes on, the same process is slowly taking place on those platforms as well. 
\begin{quote}
P20:\textit{``Some older people, some parents or grandparents only establish a Facebook account so that they can watch what their children are posting, and their children's friends are posting, they just stalk everybody."}
\end{quote}


\subsection{External Influence}
Facebook's community had a very strong devotion to the platform when it comes to staying connected. However, most users step away from social media for work or future aspirations. We wanted to explore the external influences that direct social traffic towards Facebook but also make other users dissatisfied due to the added pressure to stay online.

\subsubsection{Professional Influence}
Participants have mentioned that they are often pressured from their professional organizations to create a social media account, specifically Facebook, to have an online social presence. They have also mentioned that this enables the recruiters to keep a tab on their employees. Many employers go through their employees profile in order to keep tabs on them and participants monitor their activities accordingly.
\begin{quote}
P10:\textit{``I try and think of a job recruiter was looking at it, would they be okay with it?"}
\end{quote}
This process though help them to keep unintentional data from being leaked from the participant's account, however Facebook's metamorphosis into a social LinkedIn is something they dislike. Even school's are teaching students how to manage their social media accounts to provide a more professional representation of an individual.

\subsection{Facebook Modifications}
Facebook has modified its user experience by changing a few of its features. The changes are sometimes accepted or rejected by the users, thus imploring more or less user engagement. We found some major changes made by Facebook over time, which created discontent with its user base making the changes key points from where the users modified their behavior.

\subsubsection{Advertisements}
We noticed that users have dissatisfaction in the number of advertisements. Major concerns were expressed on the usage of personal data of the users. 
\begin{quote}
P13:\textit{``I am sometimes creeped out. If I search something, it pops out in my feed. I know they pay money to Facebook to feature their ad."}
\end{quote}

\subsubsection{Complicated Interface and Information Overload}
Participant P18 says that Facebook helps keep people together. However, this also adds to the problem of information overload and provides the user with too much information. Participants like P16 have expressed discontent of the Facebook wall to timeline transition. In their attempt to expand, Facebook has made its interface more confusing for the users to understand. 
\begin{quote}
P1:\textit{``I get turned off from multiple posts by multiples people on Facebook on the same day."}
\end{quote}

\subsubsection{Live Videos and News Feed}
Facebook has included a lot of features influenced by other social media platforms such as - Snapchat, Instagram, and others. However all these changes are not well accepted by the participants. Algorithms used to generate such lists to be viewed by the participants might not be always true and thus user specific news feeds might have it's hits and misses as proven by earlier researches~\cite{gillespie2011can}. The reason for that is they do not want the similar feature which they experience in other websites. 
\begin{quote}
P14:\textit{``I do not really see the point of having live newsfeeds... I don't really care."}
\end{quote}
If all of one's friend try to post a live news feed then it again creates the problem of information overload and is unnecessary. The participants often exclaim that Facebook should be simpler again.

\subsection{Behavioral Changes because of Facebook}
With so many people using Facebook and other social media sites as often as they do, there is bound to be behavioral changes in their everyday lives. Social media has changed the standard on many social topics. Stalking, procrastination in their work, and vocabulary are just a few of the changes that have been made to every day behavior.

\subsubsection{Internet Stalking}
The ability to be inside and still follow a person's every move, big event, and change that happens in their life is something that social media has brought to the world. Participant P10 admits to doing this, they like to know what other people are up to, and get ideas for what they should do.
\begin{quote}
P10:\textit{``Probably because I really like to know what other people are up to, what they're doing, um, kind of even in like in a nosy way, like I like to know what people are up to."}
\end{quote}

\subsection{Negative Content}
Many participants are not happy with the content shared over Facebook now-a-days; Participants complained of the harsh political critics, bad language, content like alcohol shared and in turn find no productivity from Facebook.
\begin{quote}
P3:\textit{(Examples of improper content)Underage drinking. Bad behavior vandalism. Bad words. F words, stuff like that.}
\end{quote}

\subsubsection{Politics}
Political content has been a major concern of the participants interviewed. No one wanted to see other fight over such serious issues on a social networking platform. Such content came across as harsh and a poor fit for such platform.
\begin{quote}
P18:\textit{``I hate political posts, I don't think it's the right outlet for posts like that, they're usually very biased, often times uninformed... I usually just unfollow the person if they post a political post."}
\end{quote}

\subsubsection{Ranting}
Participants also didn't welcome people ranting about their lives on Facebook. They mention that they go to Facebook to be happy, connect with family, check the updates, probably post as well; however negative content about ranting about their lives is very discouraging to like.
\begin{quote}
P19:\textit{``Some people just rant about their lives on Facebook and I don't really think that that's the correct medium for that like, if I see someone with like several paragraphs on a Facebook post, like ranting about something that's annoying them, I just scroll past it, I don't even bother looking at it because it's not worth my time to read nor do I actually care. I'll try to condense it into 140 characters and put it on twitter and that's it. That way it's three lines to look at instead of three paragraphs."}
\end{quote}

\subsection{Social Media Preference}
All social media platforms handle different tasks in different ways, which causes people to like certain platforms for different reasons. Some users prefer Facebook because Facebook combines all of the textual and non-textual information into a single, organized platform. Other users find Instagram a more visually appealing medium because it has less information and clutter than Facebook. On the other hand, P5 raises an interesting theory - that media like Snapchat and Instagram are more addictive for the current generation, who like to "live in random moments". For them, Facebook is restricted to posting "big events" while Instagram and Twitter are ideal for "everyday stories".

\subsection{Other social media}
Through the interviews, we found that many users had multiple accounts on multiple platforms. Participants have often mentioned that they prefer the snap based communication as in Instagram or Snapchat. They acknowledge the fact that Facebook is incorporating a lot of aspects of other social media websites into their own. However, this creates the problem of having a clumsy and confusing interface. Other people (P5, P13, P15, P18) like to have different uses for different platforms and keep a clear boundary between them. While P13 believes that Facebook is more `professional', P18 and P19 feel the need to have a `clean presence' on Facebook because it includes a broader relationship demographic.

\subsection{Personal Preferences}
Users use Facebook for a variety of different reasons; work, making connections, and keeping up with old friends and family are just some of these reasons. And based on these, they have different preferences when it comes to their accounts.

\subsubsection{Age Affects Use}
Many participants stated that as they got older, so did their Facebook profile in terms of maturity. Many users claimed that when they were younger they posted a lot more, but as they grew older and realized the influence your social media presence can have on your future, they decreased their production of content. Participants also mentioned the gap in age to use, how once you reach a certain age it is harder for someone to keep up on technology and how to use it. 
\begin{quote}
P2:\textit{``Teenage years tend to be a little bit more interesting. People tend to overshare a little bit more. So when I was younger, yes, I did post a lot more with status updates, thinking that people cared about what you post. And at this point in time I don't really care to share as much, since my friends list has grown and there's more acquaintances I'd like to keep in touch with but not overshare. So it's more like job or work or very occasional this is where I'll be."}
\end{quote}

\subsubsection{Profile Information}
Facebook users have the option to hide certain items of their profile from the rest of the community. Many users say that they do not utilize this feature, they do not believe the information such as their; email, birthday, city of residence, or phone number can be used against them. But other users don't allow any information to be seen except for their birth date and month. They understand the risks of not being private while on the internet.

\subsubsection{Facebook Policies}
By law, Facebook and other social media websites must have their users accept any changes to their policies before they can continue use. The issue is that users do not actually read them they just accept. This can cause many issues later on. For example, Facebook mentions the privacy is of the users, but how Facebook stores and retrieves user data is fuzzy. Although they are few, there are some users who read policies. These participants tend to keep their privacy settings high and understand the risks involved with social media.

\subsection{Privacy and Security}
This is the section we wanted to explore the most. Whether or not users understood the risks of being on social media and are taking precautions to protect themselves. Unfortunately, there were many users who though not happy trade-off between privacy, security, and Facebook use and not worry about what would happen. But there were also some that took the matter very seriously and wish to keep themselves safe.

\subsubsection{Sharing Personal Information}
Most users agreed that sharing personal information on any social media site is dangerous. P4 figured that if someone needs personal information or wants it, then they are with that user enough in person to be able to give it to them or they already know it. Most users stated that the only information that a user who did not know them would be able to pick up from their Facebook page are arbitrary things that could not put them in danger.
\begin{quote}
P5: \textit{I do have everything on lockdown, the highest privacy mode. So the only thing, say A job recruiter could see my profile pictures. They would be like, oh, regular person, since I don't post anything crazy like party pictures or anything.}
\end{quote}

\subsubsection{Information Tracking}
A few participants stated that they deactivated their Facebook account because they felt there was to much information being generated by them for others. They found it creepy and unnerving that they could be tracked by Facebook. Other users were not concerned because they felt as though if someone did want to track them, it did not matter how much security they had, it could be done. 
\begin{quote}
P7: \textit{I think that eventually all of our information will be accessed no matter what. I'm just a stubborn person pretty much. I prefer not to have that information where it can be found by people who manage to get it into the databases.  I just tried to minimize risk factors.}
\end{quote}

\subsection{No Facebook}
We want to specifically highlight that 5 of the participants did not have Facebook account when we interviewed them and 1 of the participant did not have one when she filled the survey. The reasons lie anywhere between professional pressure where an organization prohibits one to delete their account to personal choices where they do not prefer sharing data over Facebook. Participants are also noted to migrate to other social media like Snapchat. We have made some suggestions to address these issues in section~\ref{sec:implications}.
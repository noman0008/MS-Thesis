% !TEX root=nonuseinterviews.tex
\section{Conclusion}
\label{sec:conclusion}
By following a semi-structured protocol, the specific data set of the 22 interviews were analyzed. The analysis of the data yielded from the interviews helped in providing concrete examples and explanations to the findings, where we discussed on the various reasons of negative perception, reduced usage, or selective engagement of users on Facebook. It could also be seen that participants had often reported modifying their behavior by reducing their usage, rather than jumping to extremes and leaving Facebook all together. Additionally, we came up with a framework for non-use typology: feature based (users use Facebook only for particular features and abandon the rest), people based (Use Facebook to connect with a particular subset of their friend-list instead of connecting with everyone), time based (users use Facebook depending on one's availability, such as while travelling, or during ideal time and ignore the rest of the time), privacy/ideology based (users use in ways that preserve privacy..e.g. via passive consumption or self censorship), tool based (e.g., Adblock, encryption methods of messenger), and highlight based (users post only about the highlights of their life and avoid posting mundane activities or what's in their mind). There are several reasons for this behavior: Facebook Functionality Influence; Personal Preferences; Negative Perception of Facebook; and other possible Influences due to External Factors. The findings suggest some design and Facebook functionality modification, keeping in accordance with privacy concerns of the users which we discuss in Section~\ref{sec:implications}. Improved and effective communication of privacy and security features and providing the users with the right over their own data can improve user experience can refrain people from curtailing their usage from one of the most popular websites, Facebook.
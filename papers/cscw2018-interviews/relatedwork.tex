% !TEX root=nonuseinterviews.tex
\section{Related Work}
\label{sec:relatedwork}
The Internet is shaping how we communicate with each other~\cite{wellman2003social} and social media paved the road of communication~\cite{bijker2012social}. Facebook is the largest social networking website and despite criticisms, its usage continues to grow~\cite{joinson2008looking}. However, we notice users concerns regarding various features of Facebook, including its data usage policy, often moving to other social media platforms?, privacy concerns regarding posted content, advertisements, and others. These concerns lead to reduced posting, increased lurking, self-censorship, and even abandoning the platform all together~\cite{wyatt2003non,karppi2011digital,gillette2015facebook}. The creation of a Facebook quit day, where people commit to quit Facebook and never return is an example of these practices\footnote{\url{http://www.quitfacebookday.com}}. 

Technological non-use has been a matter of concern for Human Computer Interaction (HCI) scholars as it not only focuses on the design changes, it also focuses on the user experience as a whole which is a critical predictor of improved security adoption and use~\cite{baumer2015importance}. Though much work is being done on usage of social media, there are essential literature gaps in understanding the dissatisfaction in consumer base, specifically of Facebook, through qualitative research. This related work section aims to provide a comprehensive analysis of the user cited reasons through different works expressing concerns on the usage of social media, especially Facebook.

\subsection{Lurkers and Non-Usage}
Lurking is defined and interpreted differently by various researchers~\cite{crawford2009following,schultz2004lurkers} and while discussing non-use it is very important not only to discuss usage versus non-usage but also to focus on users curtailing their usage in different levels. For our study, we focus on lurkers who have curtailed their usage specifically on Facebook. We also address users who have reduced their usage and abandoned the platform entirely.

Previous studies into Facebook lurking, reduced usage, and non-usage found that there are varied contributors that lead to limiting the use of Facebook to a minimal level, deactivation of accounts, and deletion of accounts. While different factors drive these changes in the user base, one common factor which was present among the users in this study were concerns that they have regarding their privacy on Facebook~\cite{baumer2013limiting}. Nonnecke et al.~\cite{nonnecke2001lurkers}, Preece et al.~\cite{preece2004top}, and Birnholtz~\cite{birnholtz2010adopt} explored the reasons behind lurkers studying several groups over Internet through interviews, however Facebook is a larger community and warrants separate study since Facebook abandonment can arise from specific features including advertisements, moderated news feed, email notifications, etc. These studies, however, while similar in subject matter and methodology differ in the approach as well as the granularity of non-usage. 

\subsection{Privacy Concerns}
Users have become increasingly aware of privacy concerns~\cite{hargittai2010facebook}, specifically in the post-Snowden era~\cite{rainie2015americans}, and more users have switched their account to Friends Only or Custom Settings compared to the first few years of Facebook~\cite{fuchs2012political,madden2012privacy}. This shift might also be explained by a lack of granular privacy settings earlier in Facebook's history as opposed to newer privacy controls. More than half of social media users (58\%) restrict access to their profiles by customizing the privacy settings, especially women~\cite{madden2012privacy}. We noted similar results as discussed in the Section~\ref{sec:findings}.

Participants in our study also mentioned that with age the amount of disclosure reduced, similar to Nosko et al.'s work~\cite{nosko2010all}. Lampe et al. argue that privacy plays a vital role in users not joining Facebook~\cite{lampe2013users} and even leading to commit virtual (social media) suicide by leaving the platform~\cite{stieger2013commits}. Perceived privacy and security, though might be different from the actual practices preached, plays a vital role in user interaction over social media~\cite{jung2016imagined,shin2010effects,debatin2009facebook,liu2011analyzing}. These concerns are enhanced for romantic partners whose Facebook experience is made more difficult due to  omnipresence of data and the ability to track a partner's posts~\cite{gershon2011friend,madden2006online,zhao2012s}. Stories from families and friends about privacy and security concerns also play an influential role in reduction of usage by Facebook users~\cite{rader2012stories}. 

Privacy and security concerns vary among different individuals indicating privacy inequalities~\cite{litt2013understanding,kang2015my} and can have important implications in design of a system~\cite{dupree2016privacy}. What is practised and preached in privacy and security concerns vary greatly ~\cite{phelan2016s} and it becomes difficult for researchers and developers to understand the details with such granularity and specificity. As a result, a binary classification as use vs.\ non-use is unable to fully characterize non-use practices of individuals. With our study, we aim at clustering various concerns leading to different or similar attitudes and behaviors of social media users and generate a typology of non-use.

\subsection{Monitoring Old Habits}
Bauer et al., found that users desired to have the ability to go through old posts as well as modify or enhance their privacy settings ~\cite{bauer2013post}. This, often referred to as monitoring, indicates that users want to represent themselves in a way that their past posts do not justify. Self-representation is extremely important for social media users~\cite{dimicco2007identity,zhao2008identity}, thus monitoring their virtual image is of high importance as well. Our analysis, as explained in Section~\ref{sec:findings} indicates similar patterns among the users where they have expressed interest in a better way to monitor their posts and sometimes delete it.

\subsection{Audience and Other Social Media}
The audience of social media, especially Facebook, is transitioning from the younger generation to older individuals which contributes to the non-usage of Facebook. The inter generational social media usage difference is increasingly blurring~\cite{bucur1999older}. Bauer et al. suggests that users may be leaving sites like Facebook in favor of newer platforms like Snapchat. On Snapchat, a photo stays in the story only for 24 hours, and the user is notified if another user takes a screen shot of the photo~\cite{bauer2013post}. We noticed similar reactions from some of our participants as well, thus, indicating some users do not like the permanency of the posts made on Facebook. 
%However our results signify that this depends on several factors. A participant expressed using Facebook as personal media storage, indicating positive implication of such a feature.

\subsection{Abusive Posts}
Mental and physical health is often considered to be a primary source of concern. Social media, though purportedly just a medium to connect individuals, has often been noted to affect users negatively, sometimes due to usage time and sometimes due to abusive content~\cite{newyorktimes2017}. Our research showed that users, including supporters of free speech, often expressed dissatisfaction regarding abusive posts, especially regarding religious or political sentiments. Adults are however less affected as compared to teens~\cite{lenhart2011teens,rainie2012tone} by such posts, but the abusive nature of the content not only contributes to bullying over the Internet, but also leads to the discontinuation of the use of such sites.

\subsection{Social Influence}
Non-Use though a personal choice can be influenced positively or negatively by various other social and behavioral factors. Baumer et al. used a survey to study how several factors including, fear of missing out, reaction of one's social media audience impact one's perspective of not being able to leave a system~\cite{baumer2015missing}. Though survey techniques  provide an overall archival reasons for non-usage of social media platforms it does not provide the detailed analysis of what prevents one from completely abandoning such platforms. As mentioned by Baumer et. al's paper, every method including surveys to machine learning including qualitative and quantitative techniques have unique approaches to a similar problem providing more and detailed analysis of a problem~\cite{baumer2017comparing}.

Baumer et. al's other studies on details of deactivation of Facebook and reduction of usage of social media provide a mixed-methods view based on surveys to interviews. However, it demarcates the usage between group communication and whether people chose to deactivate or not~\cite{baumer2017subjects}. One's usage of social media is not binary (use and non-use) or ternary (use, limited use, and non-use). Instead, non-use may cover a wide spectrum of practices involving various levels of engagement on social media.
%Squirell in this article mentions how trolls are leading to hate speeches and even frog memes are one of the most rated hate symbols ~\cite{squirrell_2017}. 

\subsection{Miscellaneous Reasons for Lurking}
Though studies on non-usage have not been limited to Facebook, but have also covered other social media platforms, such as Twitter~\cite{schoenebeck2014giving}, varying levels of (non-)usage is still understudied. Ames et al. identified the negative effect of multitasking as one of the primary reasons for lurking on social media~\cite{ames2013managing}. Whereas, Portwood argued that the fear of misinterpretation is one of the key reasons for reducing usage on Facebook~\cite{portwood2013media}. There are arguments about the digital divide and how it prevents one from communicating similarly to a privileged individual, contributing to the avoidance of social media all together~\cite{van2005deepening,warschauer2004technology}. Hargittai explores more on the social divide by mentioning that people with more technical expertise are likely to use social media~\cite{hargittai2007whose}. Ryan and Xenos instead try to analyze the characteristic traits of individuals who use Facebook, thus indicating the negative traits by exploring the Big 5 characteristics~\cite{ryan2011uses}. Verdegem and Verhoest grouped all the traits for non-use together, mentioning inaccessibility, lack of skills, and negative perception of the technologies to be the key reasons to reduce usage~\cite{verdegem2009profiling}. Design issues have also been considered in making users cease usage of Web site~\cite{pierce2012undesigning,satchell2009beyond}. All these reasons though effective is highly scattered across various facets of social media. We apply our findings to make design recommendations for Facebook to improve its user experience (see Section~\ref{sec:implications}).
% !TEX root=nonuseinterviews.tex
\begin{abstract}
Despite the popularity of social media platforms, recent usage trends have raised questions regarding their negative effects, such as addiction, jealousy, depression, decreased well-being, invasion of privacy, reduced work productivity, cyberbullying. As a result, researchers have started paying increasing attention to the extent to which people reduce or abandon social media usage to counter the negative impacts. We conducted semi-structured interviews with 22 individuals who had indicated varying degrees of social media behavior changes in order to refine and extend the characterization of non-use and non-adoption of the most popular social media platform Facebook. In addition to confirming results from the literature, we present a framework that casts non-use as various types of \emph{deliberately selective} engagement, thus moving beyond a binary portrayal as use or non-use. Our findings hold important implications for creating more inclusive user experiences on social media platforms.
\end{abstract}
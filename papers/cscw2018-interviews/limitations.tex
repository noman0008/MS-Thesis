% !TEX root=nonuseinterviews.tex
\section{Limitations and Future Work}
\label{sec:limitations}
Our study had a few limitations. The qualitative nature of the study and personal interviews made the demographics skew mostly towards students. We tried to focus on interviewing participants who expressed reduction in their Facebook usage or do not have a Facebook account. 5 out of 22 participants did not have a Facebook account when they were interviewed. Though this proportion seems to be pretty much low, all of them had different reasons for not having a Facebook account, as discussed in the section~\ref{sec:findings}, which is quite astonishing. We want to validate the reasons by extending the study to analyze quantitative data as one of the future directions, however when we talk about user experience we aim at insightful discussions, thus, qualitative techniques should be followed ideally. While users have expressed concerns about the data usage policies of Facebook, lack of proper communication regarding the risks and the implication of such policies provide an incomplete picture to the user. There is a gap between the user's perception and the company policy practices, thus trust is a critical component determining the usage of an user. Data breaches like Cambridge Analytica pose eminent threat to user experience and should be avoided. User experience cannot be boiled down to numbers to express and thus even if the detailed interviews doesn't provide the numbers, it does provide the insights of a Facebook content producer, who are turning into consumers and lurkers. 
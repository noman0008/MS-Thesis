% !TEX root=nonuseinterviews.tex
\section{Introduction}
\label{sec:introduction}
Society has changed in the past few decades in terms of communication, especially as the influence of social media perpetually increases~\cite{ellison2007social}. The line between internet usage and social media usage is blurring with every passing day. It is theorized that by 2020, 7 out of every 10 US internet users will be connected over social media\footnote{\label{trends}\url{https://contently.com/strategist/2017/04/19/social-media-trends-2017/}}. Another aspect of social media is allowing users to communicate with free instant messaging and group chats, which is typically a better option than paying for regular cellular text messaging plans.

Facebook, one of the most popular social media platforms, launched in 2004. Having geered up to public in 2006, it has seen an explosion in popularity, reaching 1 billion users in 2015\footnote{\url{https://en.wikipedia.org/wiki/Facebook}}. It has become a worldwide phenomenon, and is now the most popular social media website. Every 60 seconds, there are 510 comments made on Facebook, 293,000 statuses made, and 136,000 photos uploaded~\cite{greenwood2016social}. As of 2016, 79\% of internet users (68\% of all U.S. adults) use Facebook~\cite{greenwood2016social}. Many other social media websites, such as Instagram, Snapchat, and Twitter have seen significant growth. As the competition increases, the ability to survive in the market becomes challenging, and many updates have been released addressing various aspects of Facebook, such as the website's interface, the company's privacy policies, the way the users interact with the site, etc. One of the reasons social media has become an integral part of many people's lives is the ability to self-represent, creating an online presence. Different social media websites have different purposes-- LinkedIn is used to obtain professional connections and social media websites such as Tinder is used as online dating services. Ellison et.al. noted that these social media websites not only help to connect, but also help in developing relationships with one's connection in both personal and professional aspects of their life~\cite{ellison2007benefits}.

Despite the many benefits of connecting on social media, users often experience extreme pressure to join a social media platform. Being a part of a virtual community is so prominent in our society, joining has become the norm and people who choose not to, may come across as odd. Therefore, it can be important for individuals who are trying to fit in to create a profile on social media. Several studies indicate that social media usage is decreasing over time. Gilette mentioned that Facebook usage has been slowly declining recently as users have been shifting to other social media platforms~\cite{gillette2015facebook}. There are articles published recently which mention how the decline of Facebook users is a gift in disguise for the privacy and security community\footnote{\url{https://www.dailydot.com/via/killing-facebook-dependency-monopoly/}}. Prior research indicates that privacy is an important feature to users when they are considering jumping in to social media. More people are realizing the risks and becoming concerned about the trade-off between privacy and staying connected with their friends. These articles are some examples of people having negative opinions about social media as \textit{agent of chaos} despite the several social benefits it provides.

In spite of it's popularity, Facebook scores low in customer satisfaction surveys, with users reporting issues with privacy, interface changes, and spam as common problems. Facebook has received much criticism in regards to its handling of user data and privacy. Sophos mentioned that businesses see Facebook as the most dangerous social media site for user security, with 60\% of businesses believing that it is more dangerous than Myspace, Twitter, and LinkedIn\footnote{\url{https://www.sophos.com/en-us/security-news-trends/security-trends/network-security-top-trends.aspx}}. Criticisms about its usage of user data has been so harsh, that in the past Facebook has been ranked in the bottom 5\% of private sector companies in user satisfaction~\cite{greenwood2016social}. In 2011, the American Customer Satisfaction Index listed Facebook as having a score of 64 out of 100, putting it in the same range as the IRS's e-filing system, airline sites, and cable companies~\cite{guynn_2014}. Researches have indicated that users are getting concerned about the negative effects of social media and how it is degrading their physical and mental health~\cite{newyorktimes2017}.

While much of the current research explores why internet users choose Facebook as their preferred social media platform, our goal was to find out why users may choose to limit their usage or abandon the website altogether and whether Facebook will follow in the footsteps of the many social media giants before them, such as Myspace.

Our study focuses on answering the following research questions:\\
\textit{RQ1. What are the factors that contribute to non-use and disengagement from social media?}\\
\textit{RQ2. What effects do privacy and security concerns have on an individual's social media behavior?}

We tackled the above research questions by conducting a pre-screening survey, where 248 participants responded. Out of the 248 responses, we selected 22 participants for the interviews who specifically mentioned they do not have Facebook account or have curtailed their usage since last year. Based on our analysis, we make the following contributions:
\begin{itemize}
    \item We describe user experience of Facebook and social media in general. Our main focus was to explore why users are curtailing or abandoning Facebook. 
    \item We uncovered the reasons why users stay connected virtually and choose to not produce any content themselves. We explored how social media modifications indicate such changes, it personal changes which can be the causative agent here. We also found out how privacy and security becomes a deciding factor when online behavior is considered.
    \item We developed an exhaustive coding scheme of the detailed interviews to create a list of reasons that can be a primary cause of user's behavior online apart from privacy concerns.
    \item We suggest changes that can be effective both from the user's and Facebook's perspective to improve user experience without compromising on their privacy.
\end{itemize}

This paper is organized as follows: in the subsequent section, we summarize prior research on the theorized reasons for user dissatisfaction of social media websites, often leading to the reduction in usage if not abandonment of the platform. We then outline the method in Section~\ref{sec:method} applied for our study along with the details of participant recruitment and a description of the sample. Next, we describe our findings in Section~\ref{sec:findings} followed by a discussion of the insight that emerged. We proceed to apply the insight to suggest a number of potential improvements to Facebook, its user experience, and other related aspects in the Section ~\ref{sec:implications}. We point out important limitations and avenues for future work in section~\ref{sec:limitations} and conclude in the section~\ref{sec:conclusion}. We have also provided the screening questions and semi-structured interview questions in the supplementary materials.

%Many people view the social media community as constantly updated status reports. 
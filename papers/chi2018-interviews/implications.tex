% !TEX root=nonuseinterviews.tex
\section{Implications and Suggestions}
\label{sec:implications}
In the introductory section~\ref{sec:introduction} of the paper, we remarked that Facebook is growing at a remarkable pace and has attained massive popularity at a global level. However, throughout the paper we have discussed multiple issues that users presently face with Facebook, and how they view and behave on the social media platform. Our findings are obtained from a survey taken by 278 users and a detailed interview of 22 participants chosen among them. The significance of this work is that we investigate peoples' behavior, their activities, and their feelings on Facebook by conducting a survey and detailed interviews. Clearly, this is an important exercise given that social media, specifically Facebook, has become an important means of communication. Such research does not only serve to improve existing social media platforms like Facebook, but also raises awareness of concerns that users experience--many of which might not be widely known or well studied. 

Unlike other studies, especially the likes of ~\cite{baumer2013limiting} where self-reported stories have been evaluated using a survey, we chose the approach of a qualitative semi-structured interview. This mode of data collection not only helped us in giving an overview of the participant's experience but also helped us in evaluating the deep rooted reasons behind why the participants are reducing their usage or leaving the website all together. We acknowledge that though Facebook does not experience any mass amount of people abandoning the site all together, the platform does still experience a decrease of users and unless the issues are addressed it can surely reach the same fate as Myspace.

\subsection{Advertisements}
The fact that all of the participants pointed out the advertisements on Facebook, suggest that something should be done. Every participant acknowledged the need for advertisement, however, if the advertisement pops up every time, this creates a major discomfort among the user. The connection of Facebook ads with their Google search or data usage over the internet is clearly recognized by the Facebook users and bearable to a certain extent--which should stay limited. No one wants someone continuously monitoring every activity, thus one of the suggestions is to reduce the number of advertisements shown to the users. This is an obvious suggestion, however when all of the participants are concerned about the particular feature then this needs to be changed.
\subsection{Privacy Settings}
Facebook frequently sends notifications that remind users to moderate Privacy settings, however there should be better ways to communicate this message. Users not only need privacy and security from the audience of their post, but also from Facebook itself and strangers. Data usage by Facebook is a serious concern among the users and there should be a mechanism in a way such that if the users allow Facebook to use their data, then only they will be able to access them. We understand every user signs the privacy policy document and Facebook wants to improve the user experience by personalizing the content; however this is creating more of negative impression as opposed to positive.
\subsection{News Feed Moderation}
Participants have acknowledged that they use News Feed, however the moderation of the News Feed though proves parsing data more interesting to the particular user, which creates an unintentional social bubble for the user. We suggest giving options to the user, whether they want to moderate their News Feed or not. If yes, they can moderate the news feed, but if not, they can choose which news to view.
\subsection{Connection Request}
Participants often complain about how the audience of Facebook has been modified to appeal the older audience. This helps the users connect more with their friends and family, however it creates a problem of an unintended audience which is undesirable. We understand users can post to a restricted list of friends over Facebook, but this should be communicated properly for users to avoid the confusion of posting to an unintentional set of viewers of the post. Participants like P17 have complained that they do not know how their Professors are going to react seeing a particular post on Facebook. We definitely cannot predict what a participant is assuming about their friends. However, the term "friends" is often misleading if one can add their friends, family, extended family, acquaintances, colleagues, and others in their friend list. Thus a better terminology for such instances can be connection-request and connections instead of friends and friend request.
\subsection{Stalking}
Stalking both from a social or professional perspective is challenging. Though as participants we cannot stop it, however, we can definitely let the users know who is searching for one's profile on Facebook, keeping similarity with Linked-In. When users are aware who is viewing their profile, this keeps them vigilant. We acknowledge this change can create negative user experience who use Facebook as a searching tool. However, stalking is a negative activity which should not be encouraged. This change can reduce several cyber-crimes as well which happen through identity theft and information theft from social media, especially Facebook.
\subsection{External Influences}
We also noticed users complaining about external influences due to which they had to curtail their usage of Facebook. In fact, P10 was an active Facebook user who explicitly mentioned to have deleted their Facebook account due to the organization where she worked. It is understandable to think about a firm and it's privacy; however, constraints should only be there to ensure that there are zero negative posts that can effect the firm, but the personal usage of an individual should remain unaffected.
\subsection{Provide Incentives}
Users often restrict their Facebook usage due to personal reasons since they do not find using Facebook as productive. This is a major concern since this is not dependent on any Facebook or external feature. However, users should be provided with proper incentives to join Facebook or keep using it. If Friends are the only reason to keep using Facebook, then removal of friends also signify removal of the person. This can create a chain reaction of users leaving Facebook, which is definitely undesirable.

% !TEX root=nonuseinterviews.tex
\section{Related Work}
\label{sec:relatedwork}

The Internet is shaping how we communicate with each other~\cite{wellman2003social} and social media in itself has paved the means of communication~\cite{bijker2012social}. Facebook is the largest social networking websites and despite several criticisms, its usage it still continues to grow~\cite{joinson2008looking}. However, we notice users complaining about their data usage policy, moving to other social media platforms, privacy concerns against their posts, and others. This often leads to reducing their usage, lurking around, and even abandoning the platform all together~\cite{wyatt2003non,karppi2011digital,gillette2015facebook}. In fact, there is a Facebook quit day, where people commit to quit Facebook to never return\footnote{\url{http://www.quitfacebookday.com}}.Though much work is being done on the social media usage, still there are some literature gaps in understanding the non-use to find the dissatisfaction in consumer base specifically of Facebook through qualitative research. This related work section aims to provide an exhaustive analysis on the reasons users have cited through different literary works expressing concerns on the usage of social media, especially Facebook. We will also mention how such studies are different in approaching and addressing the issue at hand.
\subsection{Lurkers and Non-Usage}
Lurking has different meanings and interpretation for different researchers~\cite{crawford2009following,schultz2004lurkers}. For our study, we focus on lurkers, who have curtailed their usage specifically on Facebook. We also talk about users who have reduced their usage, and abandoned the website entirely.

There have been previous studies on Facebook lurking, reduced usage, and non-usage. Baumer et al. found that there are many different contributors that lead to limiting the use of Facebook to a minimal level, deactivation of accounts, and deletion of accounts. While different reasons can lead to these changes in the user base, one common theme that was present amongst the users in this study are the concerns they have over their privacy on Facebook~\cite{baumer2013limiting}. This study however might be similar in some aspect, but the key difference lied in the approach and method of survey. They took a survey though was effective in learning how happy or contended the users were after deactivating or deleting their profile, but couldn't explicitly explore on the reasons. Apart from our analysis the suggestions made by us provide a new approach to the problem as discussed in Section~\ref{sec:implications}.

Works of Nonnecke et al.~\cite{nonnecke2001lurkers} and Preece et al.~\cite{preece2004top}also explored the reasons behind lurkers around several groups over internet through interviews, however Facebook being a larger community needs to be studied separately. Though there are some common traits with this paper, we also studied people who have abandoned Facebook for its particular functionality such as - advertisements, moderated news feed, email notifications, etc. Birnholtz discusses the reasons for adapting and then abandoning technologies such as, IM~\cite{birnholtz2010adopt}. Though it might have some similarities with Facebook's abandonment it never assures that they can be the reason. They, however, adapted the same method of interviewing as us.

All the studies mentioned above, though related to our study, do not explicitly provide detailed insights on Facebook non-usage specifically through qualitative analysis. Our work targets that gap in the literature.
\subsection{Privacy concern}
Users have become more aware of their privacy concerns in recent years, and more users have switched their account to friends only or custom settings compared to the first few years of Facebook~\cite{fuchs2012political,madden2012privacy}. About 58\% of social media users restrict access to their profiles by customizing the privacy settings, especially the women~\cite{madden2012privacy}. We noticed similar results as discussed in the Section~\ref{sec:findings}, participants also mentioned that with age, the amount of disclosure reduced, as was observed in Nosko et al.'s work~\cite{nosko2010all}. Lampe et al. also argues how privacy plays a vital role in users not joining Facebook~\cite{lampe2013users} and even committing virtual suicide by leaving the website~\cite{stieger2013commits}. Researchers discuss about how even perceived privacy and security plays a vital role in user interaction over social media~\cite{jung2016imagined,shin2010effects}. Privacy concerns arise when romantic partners are involved and they find Facebook harder to use due to vigilance and omnipresence of the website~\cite{gershon2011friend,madden2006online,zhao2012s}. Stories from families and friends about privacy and security concerns also play an influential role in reduction of usage by Facebook users~\cite{rader2012stories}.
\subsection{Monitor old habits}
Bauer et al., found that users desired to have the ability to go through old posts, and modify or enhance their privacy settings ~\cite{bauer2013post}. This often referred to as monitoring, indicates that users want to represent themselves in a way that their past posts do not justify; Self-representation is extremely important for social media users~\cite{dimicco2007identity,zhao2008identity}, thus monitoring their virtual image is of high importance as well. Our analysis, as explained in the section ~\ref{sec:findings} indicates similar patterns amongst the users where they have expressed interest in a better way to monitor their posts and sometimes delete it. %Facebook currently has a feature that allows users to view posts they made on that day at least a year ago, but with the purpose of changing the settings on the post so users can properly hide old posts that they might not want to delete. This suggests that users may tend to make posts that they will later regret. We also explored editing and deletion as a possibility for why users may not be as enthusiastic about the platform as they were in the past. 
\subsection{Audience and Other social media}
The audience of social media especially Facebook is migrating from the younger crowd to the older crowd, and this contributes to the non-usage of Facebook. Once the older crown learns the technology, they have similar usage as compared to the younger crowd~\cite{bucur1999older}. Bauer et al. suggests that users may be leaving sites like Facebook in favor of a newer platform like Snapchat. With Snapchat, a photo only stays on the story for 24 hours, and it notifies the user if another user takes a screen capture of the photo~\cite{bauer2013post}. We noticed similar reactions from our participants as well, thus, indicating some users do not like the permanency of the posts made on Facebook. However our results signify that this depends on several factors. A participant expressed using Facebook as a personal storage media, indicating positive implication of such a feature. 
\subsection{Abusive posts}
Mental and physical health is often considered to be a primary source of concern. Social media, though, seems to be just a medium to connect has often been noticed in effecting the users negatively, sometimes due to the usage and sometimes due to the abusive content shared on it ~\cite{newyorktimes2017}. Our research showed that users though supported freedom of speech often expressed dissatisfaction on abusive posts, especially hurting anyone's religious or political sentiments. Adults are however less affected as compared to teens~\cite{lenhart2011teens,rainie2012tone} by such posts but the abusive nature of the content not only contributes to bullying over internet but also leads to the discontinuation of the users over such media.
%Squirell in this article mentions how trolls are leading to hate speeches and even frog memes are one of the most rated hate symbols ~\cite{squirrell_2017}. 

\subsection{Miscellaneous reasons for lurking}
Other reasons involving negative effect of multitasking as one of the primary reasons for lurking on social media~\cite{ames2013managing} is noted by Ames. Whereas, Portwood argues the fear of misinterpretation as one of the key reasons for reducing usage on Facebook~\cite{portwood2013media}.There are arguments about the digital divide and how it does not enable one to communicate in a similar way as a privileged individual, contributes in avoiding the social media all together~\cite{van2005deepening,warschauer2004technology}. Hargittai explores more on the social divide by mentioning that people with more technical expertise are likely to use social media~\cite{hargittai2007whose}. Ryan and Xenos instead try to analyze the characteristic traits of individuals who use Facebook, thus indicating the negative traits by exploring the Big 5 characteristics~\cite{ryan2011uses}. Verdegem and Verhoest clubbed all the traits for non-using together mentioning inaccessibility, lack of skills, and negative perception of the technologies to be the key reasons to reduce usage~\cite{verdegem2009profiling}. Design issues have also been considered in making users not use the website any longer~\cite{pierce2012undesigning,satchell2009beyond}, we have suggested a few design changes for Facebook to increase or sustain their user base in Section~\ref{sec:implications}.
%In "The Political Economy of Privacy on Facebook" by Christian Fuchs, the use of Facebook user data is discussed with regards to what Facebook wants the data for. Users noted that the targeted advertisements were an obvious use of their data, since browsing history and search history on Facebook can reappear as targeted ads. In "Who Commits Virtual Identity Suicide? Differences in Privacy Concerns, Internet Addiction, and Personality Between Facebook Users and Quitters" by Stefan Stieger et al., the researchers found that Facebook quitters tend to have more concerns with regards to their privacy than the average Facebook user. Facebook quitters were also more conscious of their online profile and privacy regards. 

%Amanda Nosko noted that as Facebook users became older the amount of personal information in profiles decreased and that those profiles that were seeking a relationship were at the greatest risk of threat.  
%Nicole Ellison noted in her research that college students who have a higher rate of Facebook use have a higher level of social capital, which is against what most anecdotal evidence may suggest. Ultimately, as people move from one community to another, Facebook is a useful tool for keeping those old connections in check while also providing many opportunities to expand on their social capital. 
 
%Shanyang Zhao found that the identities of Facebook profiles were that of highly socially-desirable selves when compared to the offline, real world selves. Users on Facebook tend to construct and create their posts and profiles in a way that is desirable to others. Adam Johnson explored how people vary in social behavior on social media sites and face to face conversations. What he found was that people were a bit more self-aware on Facebook when compared to face to face conversations. People were more careful about observing how their profiles and/or posts might make them look to other people. Fundamentally, there are differences in how people interact in the real world compared to their profiles on social media sites.

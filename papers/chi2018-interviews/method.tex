%!TEX root=nonuseinterviews.tex
\section{Method}
\label{sec:method}
In order to find answers to our proposed questions we pre-screened participants through a short survey in which we had 248 responses. After narrowing down our responses we conducted a semi-structured interview on online communication practices of our participants with a focus on Facebook. In this section, we discuss how the participants were recruited and the study methodology. The sub-sections below, describe our study protocol and narrative. This study was approved by the University's Institutional Review Board (IRB).

\subsection{Recruitment}

We narrowed down our responses to 22 interviewees from several forums, including but not limited to:
Emails, social media platforms, and through mailing lists. The survey ran from December 2016 to May 2017. To avoid priming, the study was advertised without revealing the primary focus or intention of privacy and security on Facebook. The advertisement directed potential participants to a brief online screening questionnaire (Refer Supplementary Materials). The survey included several demographic questions we asked about the user's social media behavior. After the online screening questionnaire was filled out by 248 participants, those who indicated they do not have a Facebook account or have reduced their usage from the previous year were contacted to schedule interviews. We minimized cultural diversity of the sample by restricting participation to those over the age of 18, who indicated having lived in the US for at least 5 years, and were native English speakers. 

\subsection{Screening questionnaire}
Along with our restriction questions we also asked the participants questions related to their online social media behaviors such as- "Do you have a Facebook Account?, "What do you use Facebook for?", "Compared to a year ago, how often do you post on Facebook?", and "How would you characterize your use of Facebook?". Based on their given answers we recruited 22 participants, 5 of who did not have a Facebook account at the time of the interview, 16 of them have reduced their usage from the last year, and 1 participant(P10) did not have a Facebook account while filling out the screening questionnaire but re-activated her old Facebook account before the interview was conducted. 

\subsection{Interview}
The interview script was vague, giving the researcher the ability to improvise questions through out by repeated pilot studies. Please see the semi-structured interview questions in the supplementary materials for more details.

The varied range of questions to analyze the online behavioral pattern of the users included questions about other social media platforms apart from Facebook such as- "What other social media platforms do you use?", "Do you use your Facebook account to login to other services on the Internet? (Why or why not?)", etc. Furthermore the participants who have been using their Facebook account up until the interview were asked questions such as- "What prompted you to create a Facebook account?", ''How long have you had a Facebook account?", etc. We wanted to know why the participants used Facebook so we asked questions such as- "What aspects of using Facebook do you like/dislike?", "Which features of Facebook do you typically use (i.e., chat, groups, events, etc.)? Why?"

Through our set of questions we wanted to explore more on how the participants manage their online presence through their Facebook account, thus we asked questions such as- "What do you think people think of you based on your Facebook profile?
Have you ever posted anything that you later edited or deleted? Why?"
We also wanted to know why users choose to reduce usage or "lurk" on Facebook, so we asked questions such as- 
Which of the two modes (producer and consumer) is your predominant or preferred way of using Facebook? Why?"
Our study was primarily focused on trying to find whether privacy plays a role in making the users think about what to post or what to leave out, we asked questions such as- "In what ways is your use of Facebook ''public"?
What are your privacy settings on Facebook (public, friends only, only me, others) and why?
Those who didn't have a Facebook account were asked questions such as- Did you ever have a Facebook account?
If Yes, why did you choose to get rid of the Facebook account?
If No, why did you choose not to create a Facebook account?

The interviews were one-on-one interviews and took 30-45 minutes to complete, and the participants were given \$10.00 cash each as a token of our appreciation. As per the University Review Board's regulations we promised to pay the participant the compensation amount whether or not they chose to complete the study. Fortunately, none of our participants chose to drop the study in mid-way and highly interested in sharing their social media experience with us through interviews. 
\begin{table*}[h!]
\begin{tabular}{ |p{4cm}||p{4cm}|p{9cm}| }
 \hline
 Themes & Underline Categories & Example Quotes from the Interview\\
 \hline \hline
 Activity   & Do not Comment, Edited, Personalize, Tag, Track Others& P11: \textit{It has to be big events, went to things that are like a milestone of my life, freely worth to inform other people about. Other than that, if I am going to have breakfast with someone I am not going to post about it. I know a lot of people use Facebook to talk their mind but I don't think that's the platform for me. But I like to read other people's mind to read their craziness.}\\
 \hline
 Audience Influence &   Friends Only, Older People, Close-Knit, Friends Of Friends, Impression Management&P7: \textit{I think it was gradual. I enjoy producing content now as a writer producing plays in my free time. I think Facebook is not the type of audience I wanted for what I do. I'm not a high artist or anything like that.}\\
 \hline
 Influence of External Factors &School, Job, Everyone Has, Social Influence, Forced to talk& P10: \textit{I didn't have one for 6 months in the Fall, because I was part of an organization where we couldn't have any social media, but it ended in January, but it was like before that I had one and now I have one.} \\
 \hline
 Facebook Functionality &Advertisement, Facebook Modifications, Ease of Use, Facebook Games, Messenger& P7: \textit{I think that a lot of it is advertising of course, I do not like. You get the little banner ads on the side.}\\
 \hline
 Modified Behavior & Hide Posts, Passive, Reduced Usage, More Responsible, Infrequent Use& P13: \textit{I would say I post less often, just cause I don't get to know if people get to see a bunch of your posts.} \\
 \hline
 Negative Perceptions of Facebook    & Alcohol, Creepy, Blocked, Anger, Abandon&P19: \textit{Obviously using curse words, um, content that insinuates drinking or illegal activities, that type of stuff, um, things that a lot of people have conflicts over, so for something like politics, everybody has their view on politics and a lot of people don't agree, so I wouldn't post something on Facebook.}  \\
 \hline
 No Facebook    & No Facebook, No Online Communication, Never Created& P22: \textit{If I want to communicate to people I have email, um, if I want to talk to someone online if I have their email I'll shoot them an email, but if it's on a specific site like a uh, forum, then I can post a thing on a forum but I don't do that very much. I don't do much online communicating I guess.}\\
 \hline
 Non-Facebook Media Use   & Other Social Media, Migrating, Other Media, Telephone& P14: \textit{I feel like Facebook is one of the more soc- like one of the social media sites that young people and older people use, because if you go on Instagram and Snapchat and Twitter, they're more for like, the younger generations, so that's why I use it.}\\
 \hline
 Personal Influence    & Do Not Invest Time, Do Not Read Policies, Boredom, Age, Procrastinate& P2: \textit{Teenage years tend to be a little bit more interesting. People tend to overshare a little bit more. So when I was younger, yes, I did post a lot more with status updates, thinking that people cared about what you post. And at this point in time I don't really care to share as much. }\\
 \hline
 Positive feature of Facebook    & Use Actively, Producer, Trust Facebook, Branding, Appreciation& P8: \textit{I never regretted a post. If I had my old profile still open I know that I have a lot of friends that did this where their old posts from 2012 they laugh at it. If they were still available I would regret my posts from when I was really young because it was probably just song lyrics or something really cringe worthy.}\\
 \hline
 Privacy and Security    & Privacy Concern, Public, Sell Information, Uses Data, Turn Off Chat& P20: \textit{They(Facebook) want to probably make a lot of money, they use my data to probably track where, what sites I visit or what types of posts I subscribe to, they want to maybe know my spending habits, the types of movies I watch, they probably analyze the entire thing, for their benefit, um… and they just want to keep me as their customer, as a consumer, so that they can keep their billions.}\\ 
 
 \hline
\end{tabular} 
\caption{Table of Code Themes with examples quotes associated with the underline categories}
 \label{table:coding}
\end{table*}
\subsection{Transcription and Coding}
The interviews were audio-recorded, notes were taken by the researcher during the interview, and then the recording was deleted after transcription. Due to ethical concerns the data collected was anonymized and only accessed by researchers approved by the IRB. The recordings of the interviews were stored on the University servers and later deleted after the transcription was completed. 
\subsubsection{Open-Coding}
We used open-coding to explore all the avenues of the social media usage of the participants. RQDA \footnote{\url{http://rqda.r-forge.r-project.org/}}, package of R was used for coding the transcribed data. We followed the ethnographic coding scheme of open, axial, and selective coding ~\cite{glaser1968discovery} \footnote{\url{http://euroac.ffri.hr/wp-content/uploads/2011/07/Data-Analysis_Coding.pdf}}$^{,}$\footnote{\url{http://changingminds.org/explanations/research/analysis/ ethnographic_coding.htm}}. Open-coding of the semi-structured interviews helped the first author in labeling and defining the categories based on the interview data without any prior assumption of the responses. During open-coding, all of the responses were coded word for word to avoid any loss of data. For example, if the participant mentioned- \textit{I am sometimes creeped out. If I search something, it pops out in my feed. I know they pay money to Facebook to feature their ad. I am not majorly concerned but sometimes it creeps me out.}- this particular sentence was multi-coded to advertisement and privacy concern. The reason the particular response fell under both the categories was because advertisement was seen to one of the major concerns of the participants. The open coding generated 277 unique codes. After coding all the responses, the researcher moved to Axial coding, which generated a relationship and connection between all 277 codes.
\subsubsection{Selective Coding}
The open coding gave 277 underline coding, thereafter finding the related connections with axial coding we probed more on selective coding. This was both helpful for the analysis and enabled us in analyzing all the details provided by the participants via the interviews. The coding was done by the first author of the paper, thereafter the verifier observed the coding strategies and the analysis. For selective coding, we categorized them into 10 themes- the themes indicated both the factors leading to non-use of Facebook such as- Audience or Influence of External Factors and also effects such as- No Facebook or Modified Behaviour. The reason the coding was done accordingly was to segregate the non-users from the in-active users. We also wanted to learn more on whether users reduced their usage since last year to know the reasons for which their behaviour altered. 
We chose underline categories group together to create the themes, for example- Do not comment (on others posts), Edited(personal content), Tag(others or getting tagged), Track Others( tracking others, not actively posting but reading about others) were all categorized under the theme-Activity. The details of the categories is provided in Table~\ref{table:coding}. 
The following section discussed on the high level findings we derived from the interviews. 
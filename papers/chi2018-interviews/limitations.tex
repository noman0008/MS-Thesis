% !TEX root=nonuseinterviews.tex
\section{Limitations}
\label{sec:limitations}
Our study had a few limitation, due to the qualitative nature of the study and personal interviews the demographics were skewed mostly towards students. We tried to keep our focus on interviewing participants who expressed reduction in their Facebook usage or do not have a Facebook account out of which 5 participants who did not have a Facebook account. Though this number seems to be low however, all of them had different reasons for not having a Facebook account, as discussed in the section~\ref{sec:findings}, which is quite astonishing. We want to validate the reasons by extending the study to study quantitative data as one of the future directions, however when we talk about user experience aiming at insightful discussions, qualitative methods are the best to follow. While users have expressed concerns about the data usage policies of Facebook however, lack of proper communication and implication of such policies indicates that privacy is not the major deciding factor in determining personal usage of the social media website, which though surprising is understandable. This is again proved by those who are migrating to other social media websites, thus they are concerned on specifically Facebook rather than the data usage. Those who choose not to have any social presence choose one-one communication.

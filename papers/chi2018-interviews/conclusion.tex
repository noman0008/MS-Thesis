% !TEX root=nonuseinterviews.tex
\section{Conclusion}
\label{sec:conclusion}
%we analyzed a specific data set of 22 interviewees by following a semi-structured protocol. This study method not only helped us in getting a generalized overview but also helped us in provide concrete examples and reasons behind every findings due to the one-one communication. We found several reasons from Facebook functionality influence to personal preferences to negative perception of Facebook to Influence of External factors and others. Often participants have reported modifying their behaviour to reducing their usage instead of leaving the Facebook. We conclude by suggesting some design and functionality changes by keep privacy ad usage experience of the users. Improved communication of privacy and security features and providing the users right over their own data can improve user experience and avoid people curtailing their usage from one of the most popular websites.
%While there were some common traits between our Facebook users, given the limitations of our study we cannot give a decisive reason for posting less and/or leaving the site. While our participants did discuss privacy policies and settings, there were not enough common traits to really warrant claiming why users may be posting less or leaving the platform. However, our findings align with much of the literature that exist on Facebook use and abandonment, which is reassuring for us with regards to further work on this direction. We would like to continue working on the project and uncover these traits and characteristics of current and former Facebook users. 

By following a semi-structured protocol, the specific data set of the 22 interviews were analyzed. The analysis of the data yielded from the interviews helped provide concrete examples and explanations to the findings, which was ultimately the result of the one-one communication. From the data, it could be seen that participants had often reported modifying their behavior by reducing their usage, rather than jumping to extremes and leaving Facebook all together. There are several reasons for this behavior: Facebook Functionality Influence; Personal Preferences; Negative Perception of Facebook; and other possible Influences due to External Factors. The findings suggest some design and Facebook functionality modification, keeping in accordance with privacy concerns of the users. Improved communication of privacy and security features and providing the users apt right over their own data can improve user experience and avoid people curtailing their usage from one of the most popular websites, Facebook.
%While there were some common traits between our Facebook users, given the limitations of our study we cannot give a decisive reason for posting less and/or leaving the site. While our participants did discuss privacy policies and settings, there were not enough common traits to really warrant claiming why users may be posting less or leaving the platform. However, our findings align with much of the literature that exist on Facebook use and abandonment, which is reassuring for us with regards to further work on this direction. We would like to continue working on the project and uncover these traits and characteristics of current and former Facebook users. 
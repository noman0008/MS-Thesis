% !TEX root=nonuseinterviews.tex
\begin{abstract}

%Popularity of Facebook as the premier social media platform attained a milestone in 2015 with it reaching one billion users. Over time different social networking websites have sprung up, with each media platform trying to adjust and orient themselves to their user base. In this study, we investighttps://www.sharelatex.com/project/5988d2df6bfe5a37aaf290caated why social media users are altering their online behavior, including curtailing use, lurking, and abandoning it all together. 

Social media has become an integral part of everyday interactions across the globe, with the most popular social media service Facebook claiming more than two billion monthly active users. Despite this popularity, social media usage trends have raised questions regarding negative effects such as privacy, addiction, reduced well-being, cyber-bullying, etc. In spite of this, there is relatively little research that examines the extent to which such negative experiences lead people to alter, reduce, or abandon their social media usage. To this end, we conducted in-depth, semi-structured interviews with 22 individuals who had indicated reduced or no usage of Facebook. We uncovered eight main factors contributing to reduced engagement or complete abandonment. These factors are shaped by individual characteristics and preferences, social connections and interaction norms, system operation and features of Facebook, and external influences. Our findings hold important implications for a more inclusive user experience on social media platforms. 

%Our findings highlight the tension between divergent interest of the stakeholders in the social media ecosystem and hold important implications for a more inclusive user experience on social media platforms. 



%We found many common dissatisfaction characteristics, such as- superficial communication over social media, displaying aversive behavior towards polarizing contents, dissatisfaction with the over-sharing nature of Facebook users. The subjective investigation provided certainties that individuals are decreasing their usage and contributed more on the grounds which can lie anyplace between the adjustments in Facebook itself to individual personality change.



%Social media is not just a platform for ones presence over the internet, it is largely being utilized to communicate and associate with one's friends and colleagues. Over time different online networking websites have sprung up, with each media platform trying to adjust and orient themselves to their user base, thus providing a variety of user experience; some of which are at times agreeable and others not. Various researchers are studying topics about why and how users use social media, and are investigating different aspects of web-based social networking websites. In this study, we investigated why the social media users are adjusting their utilization including lessening their use, prowling around, and abandoning it all together. Investigating the user experience regarding online usage is both muddled and critical since however individuals regularly gripe on their use still they don't relinquish the stage yet adopt a latent strategy. We additionally needed to investigate the compatibility of various social media websites and in this manner our essential concentration was on Facebook which enabled us to explore whether users are moving to other social media websites or simply forsaking all the web-based social networking websites. The popularity of Facebook as the premier social media platform on the Internet attained a milestone in 2015 with it reaching one billion users. Our point was to investigate the privacy aspect of such adjustments in the utilization design from the users perspective. We created a qualitative survey that led us to interview 22 participants to comprehend the hidden factor prompting non-utilization of online networking we especially chose those participants who communicated they don't have a Facebook account or have lessened their use in the last year. We asked them various questions regarding their privacy settings, posting habits, and usage of the social media platform. While interviewing, and analyzing their responses we found the participants displayed several common characteristics, some of which are - feeling Facebook to be an impersonal medium for communication, not having real friends in their Friends list, showing aversive behavior towards polarizing contents, dissatisfaction with the over-sharing behavior by other Facebook users. All these findings helped us in analyzing the reasons for the dissatisfaction amongst Facebook users which can be used by designers, developers and researchers in making the user experience better. Our investigation gave a noble approach since the subjective examination not just gave certainties that individuals are decreasing their utilization yet in addition investigated more on the reasons which can lie anyplace between the adjustments in the website itself to individual personality change.%/

\end{abstract}

% !TEX root=foo.tex
\section{Conclusion}
\label{sec:conclusion}
This paper aims to address an under-researched issue in the field of user interaction with technology, namely, social media \textit{Non-use}. We believe this is imperative to understand the mindset of tech-savvy non-participants and the factors that play a pivotal role in this regard. We collected user generated comments from a technology blog called Slashdot and through qualitative coding got significant responses related to \textit{Non-use}. Further analysis unveiled several user dissatisfaction related to Facebook, for example, flaws in Facebook architecture, privacy and security concerns, personal issues etc. These results were bolstered by NLP based automated analysis and it turned out that the results of both approaches are consistent. Our findings have manifold implications for non-expert users, system designers, business analysts in designing and evaluating a reliable, fail-safe system. Besides, our work invokes several interesting questions, for example, Does availability of more data enable us to perform temporal (time-series) sentiment analysis on non-participants? How do the same techniques apply to other sources of data comprised of diverse population (layman and tech-savvy)? 
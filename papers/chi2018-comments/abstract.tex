% !TEX root=foo.tex

\begin{abstract}

Owing to its exponential rise in popularity and adoption, social media usage has received extensive research attention in the past decade. Recently, however, researchers have started to recognize the need to pay attention to deliberate \textit{Non-use} and non-adoption in order to uncover important shortcomings and usage barriers in these systems. We contribute to these efforts by investigating user-generated comments on posts related to Facebook on Slashdot, a news blog with a tech-savvy audience. We found that more than half of 1000 randomly selected reader comments indicated some form of \textit{Non-use}. A deeper look at these comments via qualitative coding and automated natural language analyses revealed Facebook system architecture and privacy concerns as key factors. Based on the findings, we derive implications regarding how expert users can surface important usage aspects that could potentially be informative and transformative in shaping usage practices and preferences of non-experts.


%Emergence of different social media sites and their continuous adjustment of various features has implications to how they are being used. Researchers have long been interested in understanding the factors which play vital role in non-use of these social media. However, most of the findings have been generated from explicit questioning of the users. For our study, we collected naturalistic unprompted responses from tech-savvy Slashdot (a technology news site) users and investigated if they are altering their usage engagement (e.g., reduction of usage, total abandonment etc.) with Facebook. We analyzed 1000 randomly scraped comments related to Facebook and through qualitative coding found more than half of the comments were related to Facebook non-usage. We observed that aspects related to Facebook, privacy concerns and personal influence were three of the eight key factors through second level of coding. Natural Language Processing based techniques also unveiled underlying user sentiments and major influences.  

%Emergence of different social media sites and their continuous adjustment of various features has implications to how they are being used. Researchers have long been interested in understanding the factors which play vital role in user interaction with these social media. However, naturalistic responses regarding \emph{'Non-use'} by tech-savvy users has not been well studied. To address this we investigated if Facebook users are altering their usage engagement (e.g., reduction of usage, total abandonment etc.) and the underlying influences that lead to this. To better understand whether individuals are altering their online behavior, we analyzed 1000 randomly scraped comments related to Facebook from Slashdot, a technology related news site. We performed a novel approach of multi-level qualitative coding on the dataset and further bolstered our findings with Natural Language Processing based analysis.

%We performed a novel approach of multi-level qualitative coding on the dataset: binary coding to filter comments with negative sentiments and multi-class coding to gather insight of the reasons. Further Natural Language Processing based analysis (sentiment analysis, topic modeling) bolstered our findings.



%Social media non-use has recently received attention. However, most of the findings have been generated from explicit questioning. Whereas we generate data from a naturalist setting. Unprompted comments:non-technology dependent. they understand it better, tech0savy sample, still engaging in non-use practices.

%tech-savy sample and naturalistic data why are tech-savy people telling they are nor using Facebook people are starting to see non-use users who are tech-averse looked at population diff popoulation and diff approach more than half were about some form of non-usefurther analysis 


 
\end{abstract}

% !TEX root=foo.tex
\section{Related Work}
\label{sec:relatedwork}

Our study of relevant literature comprised of Computer-Mediated Communication (instant messaging, forums etc.), blogs like Slashdot in particular, social media non-usage, and sentiment analysis on naturalistic data. 

\subsection{CMC, Blogs, Slashdot}
Computer-mediated communication of different forms has been studied by researchers for various purposes. The psychology and social aspects of using CMC was discussed by Kieslar et al. \cite{kiesler1984social}. Some argued that nature of internet use and CMC varies by countries \cite{wellman2003social} while some studies \cite{lange2006your} showed online discourse and flaming are by-products of CMC. As a major form of CMC different blogs and forums like Slashdot appeared and quickly became popular for their wide range of applications, such as education \cite{duffy2006use, pinkman2005using,  williams2004exploring}, politics \cite{davis2009typing, koop2009political}, participation study \cite{gaonkar2008micro}, technology \cite{khabiri2009analyzing} etc. So current research on Blogosphere is promising as researchers continue to study topics like content delivery, conversation analysis, ways to improve social media research through blogs \cite{ferdig2004content, herring2005conversations, hookway2008entering}. One study \cite{gilbert2009blogs} on whether blogs are echo chambers or not was particularly interesting because it adopted both empirical and algorithmic approach on comments from 33 of world's top blogs. This approach had similar implication and motivation as our paper.

Slashdot as a technology news site has gained popularity and attracted researchers from different fields. Kunegis et al. \cite{kunegis2009slashdot} analyzed Slashdot zoo corpus by representing it as a graph containing positive and negative endorsements. Statistical methods show Slashdot discussion threads have strong heterogeneity and self-similarity and thus invoke less controversial topics \cite{gomez2008statistical}. Slashdot, as a public sphere was analyzed for understanding the mechanism of other similar environments \cite{baoill2000slashdot, poor2005mechanisms}. Quantitative analysis on Slashdot as a network and its user comments \cite{gomez2008statistical, kaltenbrunner2007description} is not well explored yet. Lampe et al. \cite{lampe2007follow} tried to analyze how users filter out comments leveraging the comment rating feature of Slashdot. Characteristics of sample posters, typical post distribution, and topic categories in Slashdot were studied by Halavais \cite{halavais2001slashdot}.


\subsection{Social Media Non-use}
A lot of previous work addressed lurking and non-usage \cite{ames2013managing, crawford2009following, portwood2013media} on online forums, mailing lists, however, they are now obsolete because of change in technology and social aspects \cite{nonnecke2001lurkers}. According to Mason \cite{mason1999issues}, lurkers do not feel competent enough to post in social media. This requires extensive study (both quantitative and psychological) on non-usage to improve community experience \cite{preece2004top, schultz2004lurkers}.  Researchers have studied the necessity to understand how interventions can influence the known determinants of IT adoption and use. Baumer et al. \cite{baumer2013limiting} contribute to developing an understanding of the
sociological processes of determining what technologies are
(in)appropriate and in which contexts. Their qualitative study focuses on those users who left Facebook and investigates the sociological process of \textit{Non-use}. Comparison of users and non-users, inhibiting factors of social media use was analyzed by some work \cite{hargittai2007whose, ryan2011uses,warschauer2004technology}. Another work \cite{baumer2014refusing} focused on refusing, limiting, and departing from technology and associated study such as theories, methods, foundational texts, and central research questions in the study of \textit{Non-use}. 

Lampe et al. \cite{lampe2013users} argued why concerns about privacy, context collapse, limited time, and channel effects are some of the reasons for not joining Facebook. Madden \cite{madden2012privacy} analyzed different privacy management on social media sites and concluded that users choose restricted privacy setting and unfriending people is increasing. Their analysis unpacked some of the aspects of user behavior related to \textit{Non-use}. 


\subsection{Sentiment Analysis and Topic Modeling}
In contrast, sentiment analysis and topic modeling on user generated content in naturalistic reviews are well researched topics. In particular, sentiment analysis in blogs \cite{godbole2007large} and social media \cite{kouloumpis2011twitter, mohammad2013nrc, pak2010twitter} is very common to understand user satisfaction and it has wide range of applications, such as product review \cite{cui2006comparative, dang2010lexicon, isah2014social,   mukherjee2012feature}, opinion mining \cite{liu2012sentiment,pang2008opinion}, politics \cite{mullen2006preliminary}, financial marketing \cite{cambria2013new} etc. LDA Topic modeling is another well studied area for social science and NLP researchers as it is very good at finding inherent topics from given corpus. We only focused on the papers that covered topic modeling on social media dataset \cite{hu2013spatial, wang2012tm}, blogs in particular \cite{paul2009cross, yano2009predicting}. 

The literature reviewed above gives us the impression that although few previous works addressed the issue of social media \textit{Non-use}, they were based on qualitative study (questionnaire or interview based) \cite{baumer2013limiting, nonnecke2001lurkers} or failed to get the reviews from expert tech freak users. In this work, we mitigate these shortcomings by analyzing expert user generated content, incorporating automated analysis and hence giving a new dimension to future work.  
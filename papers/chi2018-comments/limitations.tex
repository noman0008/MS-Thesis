% !TEX root=foo.tex
\section{Limitations}
We randomly sampled 1000 comments and tried to answer our questions from those comments only. It is arguable that if these comments are representative of the whole mass or not. The whole point of random sampling was to get unrelated and unbiased data. We acknowledge the fact that, there are more (in terms of social media non-use motives) than what we have discovered due to the lack of sample. While analyzing the data we noticed that the comments often involve multiple words including positive and negative words, however the sarcastic tone altered the meaning completely. It was easier for the human coders to understand such sentiments but not for automated tools. The lack of domain specific sentiment classified dataset affected our sentiment analysis to some extent. Although VADER performs pretty well for social media data, we are not convinced if it takes the whole context of a text into account. Also, the topics generated from LDA are prone to statistical co-occurrence and hence fail to capture deep linguistic themes, but this is an universally accepted shortcoming for any probabilistic model.

%Our findings engender interesting potential research questions: Do we get similar user sentiments towards other non-Facebook privacy sensitive media? Does incorporating more data and application of machine learning based text classification approach guarantee successful prediction? For our future work we want to explore other sources where users discuss their Facebook user experience to get a comprehensive understanding. This however, is not feasible to answer in a single paper, since everyday more and more such platforms are emerging and also analysis of a particular site provides answers to problems of specific audience. Design and implementation of a text classifier, which will significantly reduce the human effort and increase data for further training, is also left as future work. Also, we would like to extend the knowledge gathered from this study to address more generic agenda, such as overall social media (not only Facebook) sentiment mining. We want to use more hand annotated data for future analysis as it will not only increase the accuracy but also help getting our domain specific sentiment.  
%The coding part is also vulnerable to noise because of the inherent fuzziness of comments and occasional lack of context, therefore the classification fully depends on the coder's discretion. 
\label{sec:limitations}

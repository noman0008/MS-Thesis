% !TEX root=foo.tex
\section{Introduction}
\label{sec:introduction}

Emergence of technology and continuous integration of internet based services in everyday life has introduced the era of "Big Data" \cite{lohr2012age}. While such advancement has flourished user experience in many aspects, it has engendered other user related issues as well, such as privacy concern \cite{shin2010effects}, data breach etc. Such issues are important to take into account while designing secure and reliable system. Owing to privacy issues users can reduce and even abandon further engagement with the system \cite{madden2012privacy} which makes this a burning question for system developers, UX designers, and advertisers alike. The proliferation of social media sites and corresponding user interaction thus invoke issues that might lead to user dissatisfaction, and lack of usage. In this paper we call such an event as \emph{Non-use}. Factors that contribute to \emph{Non-use} can threaten the sustainability and success of these systems. Therefore, a comprehensive understanding of non-usage from technical and sentimental point of view is required for building a successful ecosystem of user base.

% However, prior research on those who reduce the usage, refrain from using or totally abandon a system has not been exhaustive.  In this study, therefore, we want to unveil individual, social, and technical factors that underlie non-adoption, abandonment, and non-usage of a system. In particular, we want to uncover the barriers of privacy and security concerns, socio-economic and individual issues that results in reduced engagement. In this paper, we will investigate what factors lead to resisting, dropping out or lurking. Some of the factors that often lead to such behaviour are often- taking 'vacation' from social media, minimize their wasting time, privacy concerns, disliking a particular feature etc.

Although prior research focused on adoption of technologies and various benefits of usage \cite{joinson2008looking}, little has been studied on those who reduce the usage, refrain from using or totally abandon a system. Few studies attempted to address this issue by explicit questioning of regular users \cite{baumer2013limiting, nonnecke2001lurkers}. To fill in this gap we wanted to investigate this scarcely studied user population \cite{baumer2014refusing} and user generated content from a tech-savvy user environment. From a wide range of choices we picked Slashdot\footnote{\url{https://slashdot.org/}} ("News for Nerds. Stuff that Matters", as they say). Slashdot is a technology-related news website, the summaries of stories and links to news articles are submitted by its own readers, and each story becomes the topic of a threaded discussion among users. It is safe to assume these users are technology geeks and we get spontaneous unprompted content from their comments. We randomly collected 1000 comments that belong to the "Facebook" category, and analyzed them with an aim to answer the following research question:

\begin{quote}
    \textit{RQ1: Are expert users actually rejecting technology (Facebook in our case)?} (By rejecting we mean reduced or passive usage, dropping out, and resistance towards Facebook.)

\end{quote}
% To collect and analyze people's view and opinion regarding \emph{'Non-use'} we consulted various social media forum based sites and picked \textit{Slasdhot}. Slashdot is a technology-related news website, the summaries of stories and links to news articles are submitted by its own readers, and each story becomes the topic of a threaded discussion among users. Earlier studies though focused on non-usage of Facebook, didn't focus on the particular technology friendly group of users who interact heavily online. The active discussion in \textit{Slasdhot} not only provided a means to unveil the reasons behind the usage to non-usage transition of an individual in social media but also helped us analyze of different contexts on which the comments were made. 

 We used qualitative binary coding to determine the comments related to \emph{Non-use}. Two coders individually coded the randomly selected scraped comments of Slashdot. The method of randomization was used to avoid any form of biases which can occur to collect the data during a particular time-period. Later, on the comments coded as \emph{Non-use} by either one of the coders, we did extensive qualitative coding  to probe more on the reasons of Facebook non-usage. In this paper, we investigated what reasons (privacy and security concerns, socio-economic and individual issues etc.) lead to resisting, dropping out or lurking. In particular the focus was on individual, social, and technical factors that underlie non-adoption, abandonment, and non-usage of a system. 
 
 \begin{quote}
    \textit{RQ2: What factors underlie social media non-usage and non-adoption by techies? Can state of the art NLP techniques provide an approximation of these factors?}

\end{quote}

Our aim in this study was also to augment the findings of manual coding with automated analysis and to provide a generalized quantitative result in regards to non-usage. Thus Natural Language Processing (NLP) based techniques were used for sentiment analysis, topic modeling, and wordcloud generation on selected comments. We observed that aspects related to Facebook, privacy concerns, and personal preference were 3 of the 7 key factors responsible for \emph{Non-use}. In fine, the main contribution of this paper is an extension of methods used to gain insight regarding \textit{Non-use} and non-adoption. Further, we focus on the tech-proficient population who are often an early adopters of technology. \emph{Non-use} by such individuals can illuminate important design or operational shortcomings, especially for privacy and security matters that often require nuanced technical understanding. Moreover, we utilize naturalistic user-generated content to confirm and refine previous research on factors that underlie various forms of \emph{Non-use} and behaviors pertaining to \emph{Non-use}. 


In the rest of the paper, we first discuss all relevant prior work followed by presentation of our methods extensively and showing why a combination of NLP and qualitative coding made our results effective. Then we illustrate our findings based on empirical and automated study and further discuss the findings in detail. Thereafter, we argue on the practical implications for our findings in real world. We also discuss a few challenges faced during this work, limitations, and potential future works in the end. Then we conclude our study by summarizing the findings. 

%To begin with, we define non-participants of three types: \textbf{Resistors} (those who deliberately choose not to use a technology), \textbf{Dropouts} (those who limit or withdraw despite being an active user previously), and \textbf{Lurkers} (those whose usage is limited to passive consumption rather than active production of data and content). 




% !TEX root=foo.tex
\section{Implications}
\label{sec:implications}

The implications of the findings are manifold: from the perspective of researchers, users, service providers, and system designers. A common psychology in practice is people go with the trend or they tend to follow the crowd ~\cite{gilbert2009blogs}. So from a user's point of view, when other blog readers of Slashdot see such negative emotions associated with Facebook (or any other social media in general), they tend to get biased as well. This is mainly because since the Slashdot users are presumably technology geeks, their comments or reviews carry higher weight than usual. These people are early adopters and their knowledge stems from deeper understanding of technology. So if they are promoting \textit{Non-use}, decision making of non-experts in terms of social media usage is negatively affected as well. Our \textit{RQ1} results shed light on this. 

Our findings of NLP analysis (e.g., keywords from topic modeling, wordclouds) and manual coding resembled to a large extent which implies that in presence of a large dataset where manual coding is not possible, we can still accurately filter out user sentiment and underlying factors. This is significant because only inspecting 1000 comments is not sufficient and more data can generate better findings. 

Our \textit{RQ2} results have implications on the way future systems could be designed and existing systems could be evaluated. For example, an independent researcher or system designer wants to evaluate any existing blog or forum for user satisfaction level. With help of our NLP based techniques adopted in answering \textit{RQ2} or a similar but modified approach, a sentiment classifier can be built and user sentiment for that particular blog/forum can be assessed. This way product business related blog/forums could be benefited too because such blogs contain high volume unprompted reviews from users (e.g., Yelp, Amazon) and their target audience are the buyers. Our approach will enable them to understand user behavior or sentiment and adjust their business model accordingly. This knowledge can also be applied for developing a new media for communication too since traditional usability study of newly developed system is time consuming. We believe the most important implication for our \textit{RQ2} results is practical. For instance, a Facebook system analyst can go through the finding of this work to get a top level idea of what is affecting user engagement and ways to fix them. Leveraging the findings of the 7 influential factors and sentiment associated with them, it can be deduced that Facebook needs to improve their system architecture and enhance privacy and security measures. A more deeper analysis will reveal other system pitfalls and ways to overcome them. This is significant for users and developers alike because it will enable them to create an all inclusive environment that is devoid of the factors discovered in our findings. 
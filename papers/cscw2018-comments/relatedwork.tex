% !TEX root=nonusecomments.tex
\section{Related Work}
\label{sec:relatedwork}
Our study of relevant literature comprised of Computer-Mediated Communication (instant messaging, forums etc.), blogs like Slashdot in particular, technology use and \emph{Non-use} 

\subsection{CMC, Blogs, Slashdot}
Computer-mediated communication of different forms has been studied by researchers for various purposes \cite{ackerman1996answer}. The psychology and social aspects of using CMC was discussed by Kieslar et al. \cite{kiesler1984social}. Some argued that nature of internet use and CMC varies by countries \cite{wellman2003social} while some studies \cite{lange2006your} showed online discourse and flaming are by-products of CMC. As a major form of CMC different blogs and forums like Slashdot appeared and quickly became popular for their wide range of applications, such as education \cite{duffy2006use, pinkman2005using,  williams2004exploring}, politics \cite{davis2009typing, koop2009political}, participation study \cite{gaonkar2008micro}, technology \cite{khabiri2009analyzing} etc. So current research on Blogosphere is promising as researchers continue to study topics like content delivery, conversation analysis, ways to improve social media research through blogs \cite{ferdig2004content, herring2005conversations, hookway2008entering}. One study \cite{gilbert2009blogs} on whether blogs are echo chambers or not was particularly interesting because it adopted both empirical and algorithmic approach on comments from 33 of world's top blogs. This approach had similar implication and motivation as our paper.

Slashdot as a technology news site has gained popularity and attracted researchers from different fields. Kunegis et al. \cite{kunegis2009slashdot} analyzed Slashdot zoo corpus by representing it as a graph containing positive and negative endorsements. Statistical methods show Slashdot discussion threads have strong heterogeneity and self-similarity and thus invoke less controversial topics \cite{gomez2008statistical}. Slashdot, as a public sphere was analyzed for understanding the mechanism of other similar environments \cite{baoill2000slashdot, poor2005mechanisms}. Quantitative analysis on Slashdot as a network and its user comments \cite{gomez2008statistical, kaltenbrunner2007description} is not well explored yet. Lampe et al. \cite{lampe2007follow} tried to analyze how users filter out comments leveraging the comment rating feature of Slashdot. Characteristics of sample posters, typical post distribution, and topic categories in Slashdot were studied by Halavais \cite{halavais2001slashdot}.


\subsection{Technology Non-use}
The study and approach of human relationship with technology is required to develop a stringent understanding of use and \emph{Non-use} \cite{ baumer2015study}. A lot of previous work addressed lurking and non-usage \cite{ames2013managing, crawford2009following, portwood2013media} on online forums, mailing lists, however, they are now obsolete because of change in technology and social aspects \cite{nonnecke2001lurkers}. It is evident from these studies that \emph{Non-use} can not be strictly defined rather it is a complex phenomena and can take multiple forms \cite{baumer2015importance}. According to Mason \cite{mason1999issues}, lurkers do not feel competent enough to post in social media. This requires extensive study (both quantitative and psychological) on non-usage to improve community experience \cite{preece2004top, schultz2004lurkers}.  Researchers have studied the necessity to understand how interventions can influence the known determinants of IT adoption and use. Baumer et al. \cite{baumer2013limiting} contribute to developing an understanding of the sociological processes of determining what technologies are
(in)appropriate and in which contexts. Their qualitative study focuses on those users who left Facebook and investigates the sociological process of \textit{Non-use}. Comparison of users and non-users, inhibiting factors of social media use was analyzed by some work \cite{hargittai2007whose, ryan2011uses,warschauer2004technology}. Another work \cite{baumer2014refusing} focused on refusing, limiting, and departing from technology and associated study such as theories, methods, foundational texts, and central research questions in the study of \textit{Non-use}. Madden \cite{madden2012privacy} analyzed different privacy management on social media sites and concluded that users choose restricted privacy setting and unfriending people is increasing. Their analysis unpacked some of the aspects of user behavior related to \textit{Non-use}. Social media reversion, where users engage in deliberate \emph{Non-use} but later decide to continue using was studied by Baumer et al. \cite{baumer2015missing}. They examined survey responses by people who left Facebook for 99 days but returned sooner. They showed some significant factors affecting the reversion process, such as prior use of Facebook, privacy and surveillance, use of other social media etc. 

For better understanding of \emph{Non-use} and \emph{NS} different strategies have been adopted so far, such as questionnaire \cite{hampton2015social, baker2011their}, interview\cite{turan2013reasons} and survey \cite{baumer2015missing, brody2018opting}. Manual analysis can be tedious for large dataset and to tackle this problem researchers have been studying how statistical machine learning based approaches such as topic modeling can be incorporated to the existing grounded theory \cite{baumer2017comparing}. Another aspect of \emph{Non-use} study is non-users' perception of social media in developing countries. Wyche and Baumer \cite{wyche2017imagined} interviewed non-users living in rural Zambia and discovered what they think of Facebook, what they think are the benefits of joining and what are the inhibiting factors for their \emph{Non-use}. Satchell and Dourish \cite{satchell2009beyond} defined varieties of \emph{Non-use}, such as \textit{lagging adoption}, \textit{active resistance}, \textit{disenchantment}, \textit{disenfranchisement}, \textit{displacement} and \textit{disinterest}. They argued HCI mostly studies "use" rather than \emph{Non-use} because it is more visible and showed scenarios where study of \emph{Non-use} might be useful.

The study of \emph{Non-use} is not only limited to social media rather it has been more generic, such as (non)use of internet and information & communication technologies (ICT) in everyday life \cite{selwyn2005whose, selwyn2003apart, verdegem2009profiling, wyatt2003non}. By interviewing 1001 adults Selwyn et al. \cite{selwyn2005whose} tried to answer the questions such as, who is using and not using internet in everyday life and how non-users can be encouraged to make use of internet. In addition to this study, this author also examined \cite{selwyn2003apart} the established discourse of \emph{Non-use} of ICT, need for reformulation of factors for \emph{Non-use} and proposed a novel multi-layered model to address this issue. Different frameworks has been proposed so far for understanding technology use discontinuation \cite{verdegem2009profiling, wyatt2003non, oudshoorn2003users, lapointe2005multilevel, kim2009investigating, marakas1996passive, carroll2002just}. These may be political \cite{wyatt2003non, oudshoorn2003users}, personality factors \cite{lapointe2005multilevel, kim2009investigating, marakas1996passive}, functionality-need mismatch \cite{carroll2002just} etc. Based on strategies of segmentation and differentiation, a new policy framework is suggested by Verdegem and Verhoest \cite{verdegem2009profiling} to increase the ICT acceptance among those who are currently non-users.  

\subsection{CSCW and Social Media Usage Practice}

Previous works in CSCW addressed both usage practices \cite{grudin1988cscw} and \emph{Non-use}. Lampe et al. \cite{lampe2006face} described the anticipated audience of Facebook users and how their Facebook usage is shaped. Homophily in location surveillance and friendship network was studied by Guha and Wicker \cite{guha2015birds}. Lampe et al. \cite{lampe2013users} argued why concerns about privacy, context collapse, limited time, and channel effects are some of the reasons for not joining Facebook. The change of interaction, use an perception of Facebook over the course of time was studied by Lampe et al. \cite{lampe2008changes} by analyzing interview data of three different years. They showed Facebook use remains constant in this time period, however, the privacy concern changes. Baumer et al. \cite{baumer2017subjects} introduced Delphi method as measure of understanding the sentiment of social media use or \emph{non-use}.




The literature reviewed above gives us the impression that although few previous works addressed the issue of social media \emph{NS} and \textit{Non-use}; they were based on questionnaire \cite{hampton2015social, baker2011their}, interview\cite{turan2013reasons} and survey \cite{baumer2015missing, brody2018opting} based qualitative study \cite{baumer2013limiting, nonnecke2001lurkers}. In this work, we tried to analyze this issue by analyzing expert user generated content. The idea was to get impulsive responses from users without specifically asking them directed questions regarding social media \emph{NS}. To the best of our knowledge the study of tech-savvy users having social media \emph{NS} is a novel approach hence it gives a new dimension to the future work. 
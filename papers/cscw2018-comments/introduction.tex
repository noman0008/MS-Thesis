% !TEX root=nonusecomments.tex
\section{Introduction}
\label{sec:introduction}
%Since their inception in early 2000s, social network sites and social media have transformed everyday interactive practices. Today, the most popular of these sites, Facebook, claims more than 2 billion global active users. As of 2016, 79\% of Internet users (68\% of all U.S. adults) are estimated to use Facebook~\cite{greenwood2016social}. Such explosive growth and popularity has resulted in a great deal of research attention toward the use and positive impacts of social media in general, and Facebook in particular. For instance, it has been shown that the use of Facebook can be associated with a number of benefits, such as enhancing social connectednedness, increasing social capital, and boosting self esteem~\cite{koroleva2011its,ellison2007benefits}.

%However, more recent research indicates that increased use of social media may also lead to a number of negative effects, including addiction, feelings of jealousy, depression, decreased well-being, invasion of privacy, reduced work productivity, cyberbullying, etc. As a consequence, people have reported efforts to reduce their use of social media via tactics such as taking a ``vacation'' from Facebook or deleting their accounts. For instance, recent Facebook scandal that exposed large-scale harvesting of its user data by the British firm Cambridge Analytica resulted in the trending hashtag \#deleteFacebook.

%In the past few years, researchers have recognized the need to investigate such ``non-use''~\cite{baumer2014refusing}. However, barring a couple of notable exceptions~\cite{baumer2013limiting,lampe2013users}, research on non-adoption, non-use, and abandonment of Facebook has received disproportionately little research attention. Moreover, as Facebook functionality and policies change and people's knowledge and experience regarding Facebook evolve, initial findings regarding non-use may need to be updated correspondingly. To this end, we formulated the following research question:





Emergence of technology and continuous integration of internet based services in everyday life has introduced the era of "Big Data" \cite{lohr2012age}. While such advancement has flourished user experience in many aspects, it has engendered other user related issues as well, such as privacy concern \cite{shin2010effects}, data breach etc. Such issues are important to take into account while designing secure and reliable system. Owing to privacy issues users can reduce and even abandon further engagement with the system \cite{madden2012privacy} which makes this a burning question for system developers, UX designers, and advertisers alike. The proliferation of social media sites and corresponding user interaction thus invoke issues that might lead to Negative Sentiment \emph{(NS)}, and lack of usage (\emph{Non-use}). Factors that contribute to \emph{NS \& Non-use} can threaten the sustainability and success of these systems. Therefore, a comprehensive understanding of non-usage from technical and sentimental point of view is required for building a successful ecosystem of user base.

%However, prior research on those who reduce the usage, refrain from using or totally abandon a system has not been exhaustive.  In this study, therefore, we want to unveil individual, social, and technical factors that underlie non-adoption, abandonment, and non-usage of a system. In particular, we want to uncover the barriers of privacy and security concerns, socio-economic and individual issues that results in reduced engagement. In this paper, we will investigate what factors lead to resisting, dropping out or lurking. Some of the factors that often lead to such behaviour are often- taking 'vacation' from social media, minimize their wasting time, privacy concerns, disliking a particular feature etc.

Although prior research focused on adoption of technologies and various benefits of usage \cite{joinson2008looking}, little has been studied on those who possess \emph{NS} about and eventually reduce the usage, refrain from using or totally abandon a system \cite{wyatt2003non}. Few studies attempted to address this issue by explicit questioning of regular users \cite{baumer2013limiting, nonnecke2001lurkers}. To fill in this gap we wanted to investigate this scarcely studied user population \cite{baumer2014refusing} and user generated content from a tech-savvy user environment. From a wide range of choices we picked Slashdot\footnote{\url{https://slashdot.org/}} ("News for Nerds. Stuff that Matters", as they say) and Schneier on Security\footnote{\url{https://www.schneier.com/}}. Slashdot is a technology-related news website, the summaries of stories and links to news articles are submitted by its own readers, and each story becomes the topic of a threaded discussion among users. Bruce Schneier's blog is a collection of article/stories, hacks and latest news on Information Security where privacy experts discuss the state-of-art privacy and security concern on latest technology. It is safe to assume these users are technology geeks and we get spontaneous unprompted content from their comments. The motivation behind using these two different sources of data is to get better insight about Facebook haters \& non-users and compare the findings with an assumption that they will agree or disagree to some extent. We randomly collected 2000 comments that belong to the "Facebook" category from Slashdot and 1000 comments from Schneier's blog, and analyzed them with an aim to answer the following research question:


\begin{quote}
    \textit{RQ1: Do expert users possess negative sentiment towards technology and are actually rejecting it (Facebook in our case)?} (By rejecting we mean reduced or passive usage, dropping out, and resistance towards Facebook.)

\end{quote}
% To collect and analyze people's view and opinion regarding \emph{'Non-use'} we consulted various social media forum based sites and picked \textit{Slasdhot}. Slashdot is a technology-related news website, the summaries of stories and links to news articles are submitted by its own readers, and each story becomes the topic of a threaded discussion among users. Earlier studies though focused on non-usage of Facebook, didn't focus on the particular technology friendly group of users who interact heavily online. The active discussion in \textit{Slasdhot} not only provided a means to unveil the reasons behind the usage to non-usage transition of an individual in social media but also helped us analyze of different contexts on which the comments were made. 

 In general. we used qualitative binary coding to determine the comments related to Facebook Negative Sentiment \emph{(NS)} and \emph{Non-use}. Two coders individually coded the randomly selected scraped comments of Slashdot and Schneier's blog. The method of randomization was used to avoid any form of biases which can occur to collect the data during a particular time-period. Later, on the comments coded as \emph{NS} by either one of the coders, we did extensive qualitative coding  to probe more on the reasons of Facebook \emph{NS}. We classified comments into \emph{explicit Non-use} \& \emph{implicit NS} and performed thematic coding to investigate what reasons (privacy and security concerns, socio-economic and individual issues etc.) lead to resisting, dropping out or lurking. In particular the focus was on individual, social, and technical factors that underlie the \emph{NS} and eventual non-adoption, abandonment, and non-usage of a system. 
 
 \begin{quote}
    \textit{RQ2: What factors underlie social media \textit{NS}, non-usage and non-adoption by techies?}

\end{quote}

We presented the extent to which expert users are expressing their concern towards Facebook in two different blogosphere and how their opinions vary. This is extremely important since the underlying factors might shape the way the future systems are designed by engendering future research in the concern areas. We observed that \textit{privacy \& security concerns}, aspects related to Facebook (\textit{user experience}), and \textit{personal preference} were 3 of the 10 key factors responsible for \emph{NS} and \emph{Non-use}. In fine, the main contribution of this paper is an extension of methods used to gain insight regarding user discontent, \textit{Non-use} and non-adoption. Further, we focus on the tech-proficient population who are often an early adopters of technology. \emph{NS} of such individuals can illuminate important design or operational shortcomings, especially for privacy and security matters that often require nuanced technical understanding. Moreover, we utilize naturalistic user-generated content to confirm and refine previous research on factors that underlie various forms of \emph{NS} and behaviors pertaining to \emph{Non-use}. To this end our contribution is unique since comments acquired from such blogs/forums are in their intended social context in realistic settings instead of interview or lab environments, where study of \emph{NS, Non-use} has been conducted previously.


In the rest of the paper, we first discuss all relevant prior work followed by presentation of our methods extensively and showing why thematic qualitative coding made our results effective. Then we illustrate our findings based on empirical study and further discuss the findings in detail. Thereafter, we argue on the practical implications for our findings in real world. We also discuss a few challenges faced during this work, limitations, and potential future works in the end. Then we conclude our study by summarizing the findings. 